%! TEX root = main.tex
We have developed two traditional CFD solvers that are useful in academic and industrial applications: one for incompressible flow with immersed bodies and one for oil overland flow.
We also have a PINN solver for Navier-Stokes equations and a shallow-water equation solver using PyTorch (a neural network library).
These solvers give us a starting point to conduct a comparative study between neural network techniques and traditional CFD solvers.
The descriptions of these solvers are in section 2.
Readers can also find preliminary results of benchmarks and V\&V (verification and validation) in the same section.

In general, this project will achieve the following high-level aims:
\begin{enumerate}[label=Aim \arabic*]

    \item Understand the feasibility of neural-network-based flow solvers in practical engineering.

        Dockhorn~\cite{dockhorn_discussion_2019} compared traditional CFD solvers and neural network solvers in limited aspects.
        We would like to extend this type of discussion.

        The feasibility of a numerical solver in real-world problems depends on the trade-off between time-cost and accuracy.
        In real-world engineering, the time-cost includes the whole simulation workflow, including pre-processing and post-processing.
        For in-house solvers, it may also include the time required for code development and maintenance.
        We would like to understand the time-cost in this sense and the accuracy of neural network solvers.

        The PINN method itself is slower than traditional CFD methods.
        However, when considering pre-/post-processing, the overall time-cost may be different.
        People tend to ignore the pre-/post-processing's time-cost from solvers'.
        Actually, both labor and computational times required for pre-/post-processing highly depend on the solvers.
        For example, unstructured-grid CFD solvers require much more labor and computational times during pre-processing.
        These invisible time-costs of numerical solvers accumulate and become significant in applications such as those requiring parameter sweeps.
        So when it comes to applications like the geometry design of a fluid flow-related product, the PINN method may not be necessarily slower than traditional CFD solvers.

        By understanding the overall performance, we can better understand the role that neural network solvers can play in CFD applications.

    \item Implement and run traditional CFD solvers with neural network-specialized tools and hardware.

        When neural-network-based flow solvers are a way to adopt deep learning in CFD, another way is to implement traditional CFD solvers with neural networks' programming tools.
        One potential benefit is that such solvers should run on neural network-specialized hardware, such as TPUs.
        We already have a finite-volume shallow-water equation (SWE) solver using PyTorch.
        Using PyTorch makes the solver portable to either GPUs or TPUs.
        We would like to know how much gain we can get by running a traditional CFD solver with TPUs.
        TPU being a new technology, we have not found any CFD performance study with it against CPUs and GPUs.

\end{enumerate}
% vim:ft=tex
