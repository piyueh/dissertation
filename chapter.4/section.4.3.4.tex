%! TEX root = main.tex

Note that a steady solution might still be physical for cylinder flow at $Re=200$ in a perfect world when there is absolutely no perturbation and noise.
In the real world, the shedding is triggered by any natural perturbation.
In computer simulations, the shedding is triggered by all kinds of numerical noises, such as rounding errors and truncation errors.
These numerical noises mimic the natural perturbation.
As numerical noises also exist in PINNs, PINNs are expected to develop vortex shedding.

In traditional numerical simulations, sometimes it is also challenging to develop vortex shedding, especially if everything is symmetric.
However, we can still manually trigger shedding by using non-uniform ICs.
This did not happen to our benchmarks of PINNs, either.
The data-driven PINN can be seen as a data-free PINN with a non-uniform IC.
And given that this non-uniform IC already has vortex shedding, the vortex shedding phenomenon is supposed to continue.
Contrary to our expectation, the data-driven PINN stopped generating new vortices and returned to steady-state behavior.

The root cause for the unsteady data-free PINN's steady-state behavior is unknown to us at this point.
It is reasonable to assume the same cause also explains why the data-driven PINN falls back to non-shedding flow once we stopped feeding shedding training data.
In sections \ref{sec:pinn-2d-cylinder-re200-results} and \ref{sec:pinn-2d-cylinder-re200-koopman}, we hoped to gain more insight by examining (1) the vorticity generation and vortex transport in the vicinity of the cylinder and (2) the spectral information of the data-driven PINN.
We found that the PINN, at least the one we used in this work, is dispersive and dissipative.
Both dispersion and dissipation inhibit the oscillation and may add stability to the system.
We suspected the dispersion and the dissipation add difficulty for shedding to happen and cause the steady-state behavior.

Rahaman et al. (2019) \cite{rahaman_spectral_2019} showed that neural networks have spectral bias.
They tend to prioritize the learning of low-frequency patterns in the training data.
In the cylinder flow, the lowest frequency is $St=0$.
It is thus reasonable to suspect that the data-free PINN prioritized the learning of the mode at $St=0$, which is the steady mode, from the Navier-Stokes equations.
As for the data-driven PINN, on top of learning the solution of the Navier-Stokes equations, the PINN also had to learn from the training data with vortex shedding.
The vortex shedding in the training data forces the PINN to learn the higher-frequency data.
However, the tendency to prioritize the mode at $St=0$ made the flow quickly restore the non-oscillating solution in the Navier-Stokes equations.
And the damped secondary dynamic modes represent this restoration mechanism in the data-driven PINN.
More work is required to verify our suspicion.

In addition to the fact that the shedding does not continue after $t=140$ in the data-driven PINN, we are also not satisfied with its predictions before $t=125$.
Of course, there were no training points before $t=125$ during the training.
However, one reason for using data-driven PINNs is to have a better prediction capability on new data that were not seen before.
In other words, we expected PINNs to have a satisfying extrapolation capability.
If data-driven PINNs do not perform well on extrapolation but can only handle interpolation problems, then their advantage over regular data-only machine learning may need more scrutiny.

We must emphasize that while there are too many configurations and settings to try, our benchmark on data-driven PINNs may be very limited and naive.
Our focus in this work is still the data-free PINNs in solving PDEs.
% vim:ft=tex
