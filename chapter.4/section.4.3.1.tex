%! TEX root = main.tex
With the successful validation for the $Re=40$ cylinder flow, we now would like to test an unsteady cylinder flow at $Re=200$.

The computational spatial domain is $[-8, 25]\times[-8, 8]$, and the simulation time range is $t\in[0, 200]$.
Other problem setups are the same as those in section \ref{sec:pinn-2d-cylinder-re40-conf}, except for the kinematic viscosity.
The non-dimensional kinematic viscosity is $0.005$ to make the Reynolds number $200$.

The network architecture is the same: $(N_l, N_n)=(6, 512)$.
We also used a steady and an unsteady PINN solver on this problem, though $Re=200$ is not expected to have a steady-state solution.
The batch size and the configurations of training points are the same as the case of $N_{bs}=6,400$ in section \ref{sec:pinn-2d-cylinder-re40-conf}.
We only used one cyclical learning rate schedule: $(\eta_{low}, \eta_{high}, N_c, \gamma)=(1e-6, 1e-2, 5000, 0.9999915)$.

In addition to the steady and unsteady PINN solvers, a third case was done in this benchmark.
This case is a data-driven PINN simulation.
In this data-driven PINN case, we removed the original IC and train the PINN model against several snapshots from a PetIBM simulation.
The selected snapshots from PetIBM were those at $t=125$, $126$, $\cdots$, $139$, $140$.
These snapshots contain around $3$ full periods of vortex shedding.
In other words, we replaced the IC loss terms in equation \eqref{eq:total-residual} with  
\begin{equation}\label{eq:data-driven-loss}
    \left\{
        \begin{aligned}
            &r_{data,\vec{u}}(\Theta) = \sum\limits_{i=1}^{N_{data}}\lVert G_i^{\vec{u}} - \vec{u}_{data, i}\rVert^2 \\
            &r_{data,p}(\Theta) = \sum\limits_{i=1}^{N_{data}}\left( G_i^{p} - p_{data, i} \right)^2
        \end{aligned}
    \right.
\end{equation}
where subscript $data$ denotes the data from the PetIBM simulation.
The total number of data points from PetIBM is around $\num{17000000}$, and we only used $\num{6400}$ every iteration, meaning each data batch was repeated approximately every $\num{2650}$ iterations.
Except for replacing the IC losses with a data-driven approach, all other loss terms in \eqref{eq:total-residual} remain the same.

Note that for the data-driven case, the loss terms of PDEs and BCs were evaluated only in $t\in[125,\ 200]$ because we treated PetIBM's data as if they were ICs.
Also, because PDEs were solved even after $t=140$ (the last snapshot from PetIBM), we expected this trained model to be able to make predictions for $t>140$.

Each case ran for $\num{1000000}$ iterations with the Adam optimizer.
The hardware used was 1 NVIDIA A100 GPU for all cases.

The mentioned PetIBM simulation was done on the same K40 configuration as described in section \ref{sec:pinn-2d-cylinder-re40-conf}.
The solver configuration is the same as well, except that the grid resolution is $1485\times 720$.
% vim:ft=tex
