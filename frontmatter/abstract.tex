%! TEX root = ../main.tex
\abstract{
Solving the Navier-Stokes with deep learning has been a popular research topic since the term {\it physics-informed neural networks} (PINNs) was coined by Raissi et al. in 2017.
A search on Google Scholar using "physics-informed neural networks" as the keyword revealed a rapid growth from 60 results in 2017 to 3,340 results in 2021.
Such a blooming in PINNs raised our interest in how ready PINNs are for computational fluid dynamics applications in practical engineering.

This work aims to understand PINNs' capability to solve CFD problems from the perspectives of predictability and controllability.
Predictability helps engineering practitioners estimate the cost for the desired accuracy level, and controllability means whether practitioners can control the cost and accuracy.
We investigated convergence behaviors, scalability, cost-performance ratios, and the cost of non-computing tasks.
We also investigated several training strategies and PINNs' capability to solve flow problems with instability.

The results show that PINNs exhibit a qualitatively converging behavior with respect to model complexity.
And increasing the number of hidden layers is a better option to increase the model complexity.
However, a proper quantitative metric for evaluating model complexity is necessary for further analyses and cost-performance estimation in engineering practice.
We also found that the accuracy-loss relationship becomes more nondeterministic when the model complexity increases, which may hurt its engineering use.

On the other hand, our results show that PINNs have good weak scaling, which helps scale per-batch training points when using lower-end GPUs.
However, our benchmarks also indicate that PINNs are generally insensitive to the number of per-batch training points for unknown reasons.
Increasing per-batch training points hurts the time-to-solution for no apparent benefit---and we found PINNs' time-to-solutions were already several orders slower than conventional CFD code.

Our PINN implementation could not predict vortex shedding in a 2D cylinder flow at $Re=200$.
The unsteady PINN solver converged to a steady-state solution.
Another data-driven benchmark also converged to a steady-state solution when the predicted data were outside the time range of provided observed data.
These results cast doubt on both data-driven and data-free PINNs' capability to solve flow with instability.
Nevertheless, the PINN solver's drag coefficient predictions agreed with steady-state simulations' results.
And we additionally validated the predictions for a cylinder flow at $Re=40$.

Besides the benchmarks in PINNs, this work briefly introduces PINNs.
Starting from the difference between data-driven and data-free PINNs, we continued by introducing the history and the development of solving partial differential equations (PDEs) with data-free PINNs since 1994.
We discussed the open questions and known issues of data-free PINNs, which further motivated the present work.
In a later chapter, we detailed the mathematical expressions related to PINNs.
Moreover, we compared PINNs to conventional numerical methods and showed that PINNs is a special case of the least-square finite element methods.

We described our in-house high-performance CFD code and associated verification and validation (V\&V) to fairly compare conventional CFD methods and data-free PINNs.
The V\&V confirmed the correctness of our CFD code, and the subsequent performance benchmarks demonstrated its computational performance.
The in-house CFD code served as a baseline for PINNs' benchmarks.

In a nutshell, our results show that PINNs is a relatively new research area.
As we investigated the feasibility of PINNs, we found more open questions and challenges for PINNs to become a daily engineering tool.
While most studies in the literature focused on the theoretical capability of PINNs and toy PDEs, we hope our work can arouse others' interest in solving those challenges.
}
% vim:ft=tex