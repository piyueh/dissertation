%! TEX root = main.tex

For readers with a background in numerical methods for partial differential equations, we would like to make an analogy between traditional numerical methods and PINN.

In obtaining strong solutions to differential equations, we can describe the solution workflows of most numerical methods with five stages:

1. *Designing the approximate solution with undetermined parameters*
2. *Choosing proper approximation for derivatives*
3. *Obtaining the so-called modified equation by substituting approximate derivatives into the differential equations and initial/boundary conditions*
4. *Generating a system of linear/nonlinear algebraic equations*
5. *Solving the system of equations*

For example, to solve $\nabla U^2(x)=s(x)$, the most naive spectral method \cite{trefethen_spectral_2000} approximates the solution with $U(x)\approx G(x)=\sum\limits_{i=1}^{N}c_i\phi_i(x)$, where $c_i$ represents undetermined parameters, and $\phi_i(x)$ denotes a set of either polynomials, trigonometric functions, or complex exponentials.
Next, obtaining the first derivative of $U$ is straightforward—we can just assume $U^{\prime}(x)\approx G^{\prime}(x)=\sum\limits_{i=1}^{N}c_i \phi_i^{\prime}(x)$.
The second-order derivative may be more tricky.
One can assume $U^{\prime\prime}(x)\approx G^{\prime\prime}=\sum\limits_{i=1}^{N}c_i \phi_i^{\prime\prime}(x)$.
Or, another choice for nodal bases (i.e., when $\phi_i(x)$ is chosen to make $c_i\equiv G(x_i)$) is $U^{\prime\prime}(x)\approx \sum\limits_{i=1}^{N}c_i G^{\prime}(x_i)$.
Because $\phi_i(x)$ is known, the derivatives are analytical.
After substituting the approximate solution and derivatives in to the target differential equation, we need to solve for parameters $c_1,\cdots,c_N$.
We do so by selecting $N$ points from the computational domain and creating a system of $N$ linear equations:

\begin{equation}\label{eq:spectral-linear-sys}
    \begin{bmatrix}
        \phi_1^{\prime\prime}(x_1) & \cdots & \phi_N^{\prime\prime}(x_1) \\
        \vdots & \ddots & \vdots \\
        \phi_1^{\prime\prime}(x_N) & \cdots & \phi_N^{\prime\prime}(x_N)
    \end{bmatrix}
    \begin{bmatrix}
        c_1 \\ \vdots \\ c_N
    \end{bmatrix}
    - 
    \begin{bmatrix}
        s(x_1) \\ \vdots \\ s(x_N)
    \end{bmatrix}
    = 0
\end{equation}

Finally, we determine the parameters by solving this linear system.
Though this example uses a spectral method, the workflow also applies to many other numerical methods, such as finite difference methods, which can be reformatted as a form of spectral method.

With this workflow in mind, it should be easy to see the analogy between PINN and conventional numerical methods.
Aside from using much more complicated approximate solutions, the major difference lies in how to determine the unknown parameters in the approximate solutions.
While traditional methods solve the zero-residual conditions, PINN relies on searching the minimal residuals.
A secondary difference is how to approximate derivatives.
Conventional numerical methods use analytical or numerical differentiation of the approximate solutions, and the PINN methods usually depends on automatic differentiation.
This difference may be minor as we are still able to use analytical differentiation for simple network architectures with PINN.
However, automatic differentiation is a major factor affecting PINN's performance.

% vim:ft=tex
