%! TEX root = main.tex
The PINN solvers used in the benchmarks in chapter~\ref{chap:pinn-cases} were implemented with the help of NVIDIA's Modulus \cite{noauthor_modulus_nodate}.
Modulus utilizes symbolic expression, constructive solid geometry, computational graphs, and PyTorch to make defining a PINN PDE solver easier.
It also provides some pre-defined PDE solvers and neural network architectures.
Regular users only need to configure the computational domains and constraint instances to evaluate each loss term in equation \eqref{eq:total-residual-weighted}.
(For example, the number of training points per batch, sampling regions, or weights.)
These configurations are done with a combination of YAML files and Python code.
For simple use cases, the Python code only consists of several lines defining how to interpret the key-value pairs in the YAML files and calling APIs to set up loss terms.

Another feature of Modulus is using SymPy's symbolic expression \cite{meurer_sympy_2017} as a user interface.
For example, if users want to sample points from the region where $x > 0.5$, they add a SymPy symbolic expression instance of \lstinline{sympy.Gt(x, 0.5)} to the corresponding constraint instance.
Internally, Modulus does not rely on symbolic calculation.
Instead, it translates users' symbolic expressions to numeric calculations in PyTorch.
With this symbolic user interface, users are even allowed to define new PDEs by using symbolic expressions.
Please refer to Modulus' manual for details.

Modulus is still an experimental product at the time of writing.
It lacks documentation for code developers. 
To meet our needs for conducting benchmarks, after reading and understanding the source code of Modulus, we were able to customize and added new code that work with Modulus.
The additions and modifications associated with the present work include:
\begin{enumerate}[nolistsep]
    \item A reimplementation of the unsteady incompressible Navier-Stokes equations.
        Modulus also ships a similar class for the compressible Navier-Stokes equations, which falls back to the same incompressible formulation when the density is a constant.
        We reimplemented it for better control over the source code.
    \item Other non-governing PDEs, such as vorticity, Q criterion, and Neumann and convective boundary conditions.
        These PDEs are subclasses of \lstinline{modulus.pdes.PDES}.
        The first two PDEs are for post-processing, and the remaining two are for evaluating the BC losses.
    \item A binding to PyTorch's \lstinline{CylicLR} learning schedule.
    \item Nonlinear CG and inexact line search as a subclass of \lstinline{torch.optim.Optimizer}.
    \item Reimplementation of the annealing loss aggregation based on equation \eqref{eq:annealing-in-this-work}.
        The original annealing loss aggregation implementation was hard-coded for single-variable PDEs, which does not work with PDE systems like the Navier-Stokes equations.
    \item A solution workflow controller that allows the use of annealing loss aggregation, Adam, nonlinear CG, and SWA in an arbitrary combination. 
    \item An inferencer that allows to output only the desired portion of a PINN model every given iterations. 
    \item Boundary and interior constraint classes that are aware of the counter of training iterations so that they keep re-using the same batch of training points when the iteration counter remains the same.
    They are used when the CG optimizer and the annealing loss aggregation are both enabled.
\end{enumerate}

The code for the mentioned new/reimplemented components is available at \cite{chuang_dissertation_nodate}.
The same reference also contains all other code, scripts, and the container definition files for reproducing the results and figures in the later benchmarks.
% vim:ft=tex
