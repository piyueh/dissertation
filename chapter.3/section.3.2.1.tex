%! TEX root = main.tex

In the previous section, $G$ denotes a neural network model (or any mathematical model) that predicts flow quantities at given spatial-temporal coordinates.
In this section, we would like to introduce the mathematical expression of the most common neural network model in data-free PINNs: MLP (multilayer perceptron) networks.

The universal approximation theorem \cite{hornik_approximation_1991} states that large-enough MLP networks can approximate any smooth functions.
This theorem justifies the use of MLP networks as PDEs' approximation solutions.
These PDEs include the Navier-Stokes equations, which are the primary governing equations for CFD.\footnote[0]{Though the existence of the smooth solutions for the Navier-Stokes equations is not yet proved, as CFD practitioners, we usually ignore this theoretical fact.}

Figure \ref{fig:mlp-graph} is a commonly seen graphical illustration of how MLP works in literature.
\begin{figure}[hbt!]
    \singlespacing
    \includegraphics[width=0.95\linewidth]{figs/mlp.tikz}
    \caption{Graphical illustration of MLP networks}
    \label{fig:mlp-graph}
\end{figure}
To avoid introducing more mathematical symbols, in this section and figure \ref{fig:mlp-graph}, we reuse the notation from the previous section.

An MLP network is fundamentally a series of linear-nonlinear mapping pairs:
\begin{equation}\label{eq:mlp-formula}
    \begin{array}{ll}
        \vec{h}^0 \equiv \begin{bmatrix} \vec{x} \\ t \end{bmatrix} & \\
        \vec{h}^k = \sigma_{k-1}\left(A^{k-1}\vec{h}^{k-1}+\vec{b}^{k-1}\right)\text{,} & 1 \le k \le N_l \\
        \vec{h}^{N_l+1}\equiv \begin{bmatrix} G^{\vec{u}} \\ G^p \end{bmatrix} = \sigma_{N_l}\left(A^{N_l}\vec{h}^{N_l}+\vec{b}^{N_l}\right) &
    \end{array}
\end{equation}
$\vec{h}^k$ is called a hidden layer or a latent space vector.
$\sigma_{k}$ is a chosen element-wise nonlinear function.
$A^{k}$ and $\vec{b}^k$ for $k=0$ to $k=N_l$ are parameter matrices and vectors that consist of free model parameters.
In other words, $\Theta=\left\{A^0, \vec{b}^0, A^1, \vec{b}^1, \cdots \right\}$.
$N_l$ denotes the number of hidden layers, and $N_n$ in figure \ref{fig:mlp-graph} represents the number of elements in each hidden layer, $\vec{h}^k$.
Elements in $\vec{h}^k$ are often called neurons in deep learning.
Theoretically, $N_n$ does not have to be constant.
Each $\vec{h}^k$ can have different $N_n$.
However, as there is no clear guideline on specifying variate $N_n$ for PINNs, most reports in PINNs just used a constant $N_n$ across hidden layers.
We follow the same approach in this work.

$A^k$ and $\vec{b}^k$ are the free parameters we want to determine in the optimization step (i.e., equation \eqref{eq:total-residual-weighted}).
$N_l$ and $N_n$ are also parameters that control the complexity and accuracy of an MLP network.
However, they are determined by users rather than by optimization.
These parameters are called hyperparameters---they are not counted in the degree of freedom of a mathematical model.

The element-wise nonlinear functions $\sigma_{k}$ are also pre-determined by users.
Theoretically, they can be different functions across different $k$.
To our knowledge, most reports in PINNs used the same function for all $k$, i.e., $\sigma_0(x)=\sigma_1(x)=\cdots=\sigma(x)$.
In PINNs, the most commonly seen choices of $\sigma$ are hyperbolic tangent ($\tanh$) and the sigmoid function:
\begin{equation}
    \operatorname{sigmoid}(x) = \frac{1}{1+\exp(-x)}
\end{equation}
In this work, we use SiLU for $\sigma$ \cite{hendrycks_gaussian_2020}, which is the default option for Modulus:
\begin{equation}\label{eq:silu}
    \operatorname{SiLU}(x) = \frac{x}{1+\exp(-x)}
\end{equation}
To our best knowledge, in terms of solving flow problems with data-free PINNs, no reports have discussed the effect of $\sigma$ except for \cite{li_integration_2010}.

While simple MLP networks like \eqref{eq:mlp-formula} are not popular in modern deep learning applications, they are still the first choice in solving PDEs with data-free PINNs.
The reason may be that the universal approximation theorem proves MLP networks' ability to approximate any smooth functions, which makes them theory-backed mathematical models for PDEs' general solutions.
Other types of networks lack such a theoretical backup.
Nevertheless, lacking theoretical proof does not mean other networks would not work.
For example, \cite{sirignano_dgm:_2018} reported the success of using an LSTM-style network to solve PDEs.

Networks with higher complexity imply a more expensive computational cost.
Whether using complicated networks is worth it may be a question from the perspective of cost-performance ratios.
Instead, in this work, we used a variant of MLP networks implemented in Modulus---a weight-normalization MLP \cite{salimans_weight_2016}:
\begin{equation}\label{eq:weighted-norm-mlp-formula}
    \begin{array}{ll}
        \vec{h}^0 \equiv \begin{bmatrix} \vec{x} \\ t \end{bmatrix} & \\
        \vec{h}^k = \sigma_{k-1}\left(
            \operatorname{diag}\left(\vec{w}_g^{k-1}\right)
            \operatorname{diag}\left(\vec{w}_n^{k-1}\right)^{-1}
            W^{k-1}\vec{h}^{k-1}+\vec{b}^{k-1}
        \right)\text{,} & 1 \le k \le N_l \\
        \vec{h}^{N_l+1}\equiv \begin{bmatrix} G^{\vec{u}} \\ G^p \end{bmatrix} = \sigma_{N_l}\left(A^{N_l}\vec{h}^{N_l}+\vec{b}^{N_l}\right) &
    \end{array}
\end{equation}
$\vec{w}_g^{k}$ for $k=0,\cdots,N_l$ represents a parameter vector.
$W^k$ is a parameter matrix.
$\vec{w}_n^{k}$ is not a parameter vector but holds the norms of each row in $W^k$.
The key idea is to decouple each row in $A^k$ to a length multiplying a direction.
$\operatorname{diag}\left(\vec{w}_n^k\right)W^k$ is a row-normalized matrix, that is, each row is a unit vector.
$\vec{w}_g^k$ represents the lengths of the corresponding rows.

Compared to using other types of networks, a weight-normalization MLP \eqref{eq:weighted-norm-mlp-formula} does not actually alter the architecture of an MLP network.
It just rearranges the mathematical expression.
Theoretically, either using \eqref{eq:mlp-formula} or \eqref{eq:weighted-norm-mlp-formula}, a perfect optimization scheme should find $A^k$ and $\operatorname{diag}\left(\vec{w}_g^{k-1}\right) \operatorname{diag}\left(\vec{w}_n^{k-1}\right)^{-1} W^{k-1}$ equivalent to each other.
However, the real world is far from perfect.
Salimans and Kingma \cite{salimans_weight_2016} found that using the expression of \eqref{eq:weighted-norm-mlp-formula} speeds up the convergence when using 1st-order optimization schemes, such as the basic gradient-descent and Adam.

The weight-normalization variant of the MLP networks has the following formula to calculate the total number of free parameters:
\begin{equation}\label{eq:dof-calculator}
    DoF =
     N_{n} \left(N_{in} + 2\right) + 
    N_{n} \left(N_{n} + 2 \right) \left(N_l-1\right) +
     N_{out}\left(N_{n} + 1\right)
\end{equation}
where $DoF$ means the degree of freedom, another term for the number of free parameters.
% vim:ft=tex
