%! TEX root = main.tex
The PINN method relies on automatic differentiation to evaluate the derivatives of $G$.
Automatic differentiation algorithms record how a numeric value is calculated and then apply the chain rule to determine that value's derivatives with respect to its inputs.
For example, assume a piece of code evaluates the following calculation:
\begin{equation}\label{eq:graph-example}
 y = \left.{e^{\left(x_1x_2-\sin{\left(x_2\right)}\right)}+3x_1}\right|_{\begin{subarray}{l}x_1=1.2 \\ x_2=0.5\end{subarray}} = 4.72\dots
\end{equation}
On computers, the calculation is realized by a sequence of basic unary and binary operations on numeric values.
Automatic differentiation algorithms record such a sequence in a computational graph, as illustrated in figure~\ref{fig:automatic-differentiation-forward}.
\begin{figure}[hbt!]
    \Centering
    \begin{minipage}[c]{0.4\textwidth}
        \includegraphics[width=\linewidth]{figs/automatic-differentiation-forward.tikz}
    \end{minipage}%
    \begin{minipage}[c]{0.5\textwidth}
        \scriptsize
        \singlespacing
        \begin{equation*}
            \begin{aligned}
                &v_0 = \mathrm{assign}\left(1.2\right) \\
                &v_1 = \mathrm{assign}\left(0.5\right) \\
                &v_2 = \mathrm{assign}\left(3\right) \\
                &v_3 = \mathrm{multiply}\left(v_2, v_0\right)  = 3 \times 1.2 = 3.6 \\
                &v_4 = \mathrm{multiply}\left(v_0, v_1\right) = 1.2 \times 0.5 = 0.6 \\
                &v_5 = \sin\left({v_1}\right) = \sin\left(0.5\right) = 0.479\dots \\
                &v_6 = \mathrm{negate}\left(v_5\right) = -0.479\dots \\
                &v_7 = \mathrm{add}\left(v_4, v_6\right) = 0.12\dots \\
                &v_8 = \exp\left({v_7}\right) = 1.128\dots \\
                &v_9 = \mathrm{add}\left(v_3, v_8\right) = 4.728\dots
            \end{aligned}
        \end{equation*}
    \end{minipage}
    \caption{Computational graph of equation~\ref{eq:graph-example}}%
    \label{fig:automatic-differentiation-forward}
\end{figure}

It is relatively easy to evaluate the derivatives of unary and binary functions with respect to their inputs.
Combining this fact and the chain rule, automatic differentiation evaluates:
\begin{equation}\label{eq:reverse-graph-example}
    \left\{
        \begin{array}{ll}
            d_9 = \pdiff{v_9}{v_9} = 1 & \\
            d_i = \pdiff{v_9}{v_i}
                = \sum\limits_{p\in P(i)}\pdiff{v_9}{v_p}\pdiff{v_p}{v_i}
                = \sum\limits_{p\in P(i)}d_p\pdiff{v_p}{v_i}, & \text{for }i=8, 7, \ldots, 1, 0
        \end{array}
    \right.
\end{equation}
The set $P(i)$ denotes the parents of node $i$ in the reversed computational graph.
For example, as shown in figure~\ref{fig:automatic-differentiation-backward}, the parents of node $1$ are $4$ and $5$.
Any derivatives with respect to $v_1$ have contributions from nodes $4$ and $5$.
Note that we evaluate equation~\ref{eq:reverse-graph-example} in a reversed order, $i=9$, $8$, $7$, $\cdots$, so that $d_p$ ($\forall p\in P(i)$) are ready when evaluating any $d_i$.
See figure~\ref{fig:automatic-differentiation-backward} for the backward calculation.
\begin{figure}[hbt!]
    \begin{minipage}{0.4\textwidth}
        \includegraphics[width=\linewidth]{figs/automatic-differentiation-backward.tikz}
    \end{minipage}%
    \begin{minipage}{0.5\textwidth}
        \scriptsize
        \singlespacing
        \begin{equation*}
            \begin{aligned}
                d_9& = \pdiff{v_9}{v_9} = 1 & d_8& = d_9 \pdiff{v_9}{v_8} = 1 \\[-2pt]
                d_7& = d_8 \pdiff{v_8}{v_7} = 1.128\dots & d_6& = d_7 \pdiff{v_7}{v_6} = 1.128\dots \\[-2pt]
                d_5& = d_6 \pdiff{v_6}{v_5} = -1.128\dots & d_4& = d_7 \pdiff{v_7}{v_4} = 1.128\dots \\[-2pt]
                d_3& = d_9 \pdiff{v_9}{v_3} = 1 & d_2& = d_3 \pdiff{v_3}{v_2} = 1.2
            \end{aligned}
        \end{equation*}%
        \begin{equation*}
            \begin{aligned}
                d_1& = d_4 \pdiff{v_4}{v_1} + d_5 \pdiff{v_5}{v_1} = d_4v_0 + d_5\cos(v_1) = 0.36\dots = \pdiff{y}{x_2} \\
                d_0& = d_4 \pdiff{v_4}{v_0} + d_3 \pdiff{v_3}{v_0} = d_4v_1+d_3v_2 = 3.56\dots = \pdiff{y}{x_1}
            \end{aligned}
        \end{equation*}
    \end{minipage}
    \caption{Computational graph of automatic differentiation of equation~\ref{eq:graph-example}}%
    \label{fig:automatic-differentiation-backward}
\end{figure}

Given that $v_9=y$, $v_0=x_1$ and $v_1=x_2$, once the backward calculation reaches $i=0$ and $i=1$, we automatically obtain $\pdiff{y}{x_1}$ and $\pdiff{y}{x_2}$.
Please refer to reference~\cite{griewank_automatic_1988} for a detailed introduction to automatic differentiation algorithms.

No matter how complicated a mathematical expression is (like a neural network $G$), it is always broken down to a sequence of simple binary or unary operations.
And hence automatic differentiation guarantees the exact derivatives in terms of this sequence of unary and binary computation.

Automatic differentiation is used to evaluate the derivatives of $G$'s outputs with respect to its inputs.
Also, it is used to evaluate the derivatives of the residuals with respect to model parameters, i.e., $\nabla_{\Theta} r(\Theta)$.
The latter is used by optimizers in the optimization process.
% vim:ft=tex
