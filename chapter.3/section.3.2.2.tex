%! TEX root = main.tex

When dealing with periodic boundary conditions, to calculate the loss (i.e., $\vec{r}_{bc,\vec{u}}$ and $\vec{r}_{bc, p}$), one possible approach is to define the loss terms as the difference in the values between the pair of the corresponding boundaries.
For example, assume we have a periodic BC on $x=0$ and $x=L$.
We define
\begin{equation}\label{eq:naive-periodic-bc}
    \begin{aligned}
    &r_{bc,\vec{u}}(\Theta) = \sum\limits_{i=1}^{N_{BC}} \lVert G^{\vec{u}}(\vec{x}_{0,i}, t_i; \Theta) - G^{\vec{u}}(\vec{x}_{L,i}, t_i; \Theta) \rVert^2\\
    &r_{bc,p}(\Theta) = \sum\limits_{i=1}^{N_{BC}} ( G^{p}(\vec{x}_{0,i}, t_i; \Theta) - G^{p}(\vec{x}_{L,i}, t_i; \Theta) )^2
    \end{aligned}
\end{equation}
where $\vec{x}_{0,i}$ and $\vec{x}_{L,i}$ are spatial coordinates on the plane of $x=0$ and $x=L$, respectively.
And coordinate components of $y$ and $z$ in $\vec{x}_{0,i}$ and $\vec{x}_{L,i}$ should match.

Modulus provides an alternative approach.
Consider a pair of periodic BCs on $x=x_1$ and $x=x_2$.
We notice that $\sin(2\pi\frac{x-x_1}{x_2-x_1})$ and  $\cos(2\pi\frac{x-x_1}{x_2-x_1})$ have a period of $x_2-x_1$.
Hence, we expand the inputs of $G$ to
\begin{equation}\label{eq:periodic-G}
    G = G(\sin(2\pi\frac{x-x_1}{x_2-x_1}), \cos(2\pi\frac{x-x_1}{x_2-x_1}), y, z, t; \Theta)
\end{equation}
If component $y$ or $z$ also has periodic BCs, they are converted as well, following the same logic.
In the present work, the periodic BCs are handled using this approach.
This approach builds the periodicity into the model directly; hence no BC loss terms are needed for periodic BCs.
% vim:ft=tex