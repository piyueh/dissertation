%! TEX root = main.tex

The basic form of the PINN method (\cite{raissi_physics-informed_2019,cai_physics-informed_2021}) starts from approximating $\vec{U}$ and $p$ with a neural network:
\begin{equation}\label{eq:neural-network}
    \begin{bmatrix}
        \vec{U} \\ p
    \end{bmatrix}(\vec{x}, t)
    \approx
    G(\vec{x}, t; \Theta)
\end{equation}
Here we use a single network that predicts both pressure and velocity fields.
It is also possible to use different networks for them separately.
Later in this work, we will use $G^U$ and $G^p$ to denote the predicted velocity and pressure from the neural network.
$\Theta$ at this point represents the free parameters of the network.

To determine the free parameters, $\Theta$, ideally, we hope the approximate solution gives zero residuals for equations (\ref{eq:continuity}), (\ref{eq:momentum}), and (\ref{eq:ic-and-bc}).
That is

\begin{equation}\label{eq:residuals}
    \begin{aligned}
        & r_{1}(\vec{x}, t; \Theta) \equiv \nabla \cdot G^{U} = 0 \\
        & r_{2}(\vec{x}, t; \Theta) \equiv \frac{\partial G^{U}}{\partial t}+(G^{U} \cdot \nabla) G^{U}+\frac{1}{\rho} \nabla G^p -\nu \nabla^{2} G^{U} - \vec{g} =0 \\
        & r_{3}(\vec{x}; \Theta) \equiv G^{U}_{t=0}-\vec{U}_0 = 0 \\
        & r_{4}(\vec{x}, t; \Theta) \equiv G^{U}-\vec{U}_\Gamma = 0,\ \forall \vec{x} \in \Gamma \\
        & r_{5}(\vec{x}, t; \Theta) \equiv G^{p}-p_\Gamma = 0,\ \forall \vec{x} \in \Gamma \\
   \end{aligned}
\end{equation}

And the set of desired parameter, $\Theta=\theta$, is the common zero root of all the residuals.

The derivatives of $G$ with respect to $\vec{x}$ and $t$ are usually obtained using automatic differentiation. 
Nevertheless, it is possible to use analytical derivatives when the chosen network architecture is simple enough, as reported by early-day literature \cite{lagaris_artificial_1998,Li2003}.

If residuals in (\ref{eq:residuals}) are not complicated, and if the number of the parameters, $N_\Theta$, is small enough, we may numerically find the zero root by solving a system of $N_\Theta$ nonlinear equations generated from a suitable set of $N_\Theta$ spatial-temporal points.
However, the scenario rarely happens as $G$ is usually highly complicated and $N_\Theta$ is large.
Moreover, we do not even know if such a zero root exists for the equations in (\ref{eq:residuals}).

Instead, in PINN, the condition is relaxed.
We do not seek the zero root of (\ref{eq:residuals}) but just hope to find a set of parameters that make the residuals sufficiently close to zero.
Consider the sum of the $l_2$ norms of residuals:

\begin{equation}\label{eq:total-residual}
    r(\vec{x}, t; \Theta=\theta) \equiv \sum\limits_{i=1}^{5} \lVert r_i(\vec{x}, t; \Theta=\theta) \rVert^2,\ \forall \left\{\begin{array}{l}x \in \Omega \\ t\in[0, T]\end{array}\right.
\end{equation}

The $\theta$ that makes residuals closest to zero (or even equal to zero if such $\theta$ exists) also makes (\ref{eq:total-residual}) minimal because $r(\vec{x}, t; \Theta) \ge 0$.
In other words,

\begin{equation}\label{eq:objective}
    \theta = \operatorname*{arg\,min}\limits_{\Theta} r(\vec{x}, t; \Theta)\,\ \forall \left\{\begin{array}{l}x \in \Omega \\ t\in[0, T]\end{array}\right.
\end{equation}

This poses a fundamental difference between the PINN method and traditional CFD schemes, making it potentially more difficult for the PINN method to achieve the same accuracy as the traditional schemes.
We will discuss this more in section 3.
Note that in practice, each loss term on the right-hand-side of equation (\ref{eq:total-residual}) is weighted.
We ignore the weights here for demonstrating purpose. 

To solve (\ref{eq:objective}), theoretically, we can use any number of spatial-temporal points, which eases the need of computational resources, compared to finding the zero root directly.
Gradient-descent-based optimizers further reduce the computational cost, especially in terms of memory usage and the difficulty of parallelization.
Alternatively, Quasi-Newton methods may work but only when $N_\Theta$ is small enough.

However, even though equation (\ref{eq:objective}) may be solvable, it is still a significantly expensive task.
While typical data-driven learning requires one back-propagation pass on the derivatives of the loss function, here automatic differentiation is needed to evaluate the derivatives of $G$ with respect to $\vec{x}$ and $t$.
The first-order derivatives require one back-propagation on the network, while the second-order derivatives present in the diffusion term $\nabla^2 G^U$ require an additional back-propagation on the first-order derivatives' computational graph. 
Finally, to update parameters in an optimizer, the gradients of $G$ with respect to parameters $\Theta$ requires another back-propagation on the graph of the second-order derivatives.
This all leads to a very large computational graph.
We will see the performance of the PINN method in the case studies.

In summary, when viewing the PINN method as supervised machine learning, the inputs of a network are spatial-temporal coordinates, and the outputs are the physical quantities of our interest.
The loss or objective functions in PINN are governing equations that regulate how the target physical quantities should behave. 
The use of governing equations eliminates the need for true answers.
A trivial example is using Bernoulli's equation as the loss function, i.e., $loss=\frac{u^2}{2g}+\frac{p}{\rho g}-H_0+z(x)$, and a neural network predicts the flow speed $u$ and pressure $p$ at a given location $x$ along a streamline.
(The gravitational acceleration $g$, density $\rho$, energy head $H_0$, and elevation $z(x)$ are usually known and given.)
Such a loss function regulates the relationship between predicted $u$ and $p$ and does not need true answers for the two quantities.
Unlike Bernoulli’s equation, most governing equations in physics are usually differential equations (e.g., heat equations).
The main difference is that now the PINN method needs automatic differentiation to evaluate the loss.
Regardless of the forms of governing equations, spatial-temporal coordinates are the only data required during training.
Hence, throughout this paper, training data means spatial-temporal points and does not involve any true answers to predicted quantities.
(Note in some literature, the PINN method is applied to applications that do need true answers, see \cite{cai_physics-informed_2021}. These applications are out of scope here.)

\begin{equation}\label{eq:silu}
    \operatorname{SiLU}(x)=\frac{x}{1+\exp({-x})}
\end{equation}

% vim:ft=tex
