%! TEX root = main.tex
\chapter{PINNs and Details of the Methods Used in This Work}\label{chap:pinn}

\section{Solution Workflow}\label{sec:pinn-workflow-overview}

    \subsection{Overview}\label{sec:pinn-overview}
    %! TEX root = main.tex
Consider the incompressible Navier-Stokes equations in a spatial domain of $\vec{x}\in\Omega$ and a time range of $t\in[0$, $T]$:
\begin{subequations}\label{eq:orig-ns}
    \begin{empheq}[left=\left\{\,, right=\right.]{align}
        &\pdiff{\vec{u}}{t} + \left(\vec{u} \cdot \nabla\right) \vec{u}
            =
            -\frac{1}{\rho}\nabla p + \frac{1}{Re} \nabla^2 \vec{u}
            \label{eq:orig-ns-momentum} \\
        &\nabla \cdot \vec{u} = 0 \label{eq:orig-ns-cont}
    \end{empheq}
\end{subequations}
A solution to the Navier-Stokes equations is subject to IC and BCs:
\begin{equation}\label{eq:orig-ns-ic}
    \left\{
        \begin{array}{l}
            \vec{u}(\vec{x}, t=0) = \vec{u}_0(\vec{x}) \\
            p(\vec{x}, t=0) = p_0(\vec{x}) \\
        \end{array}
    \right.
    \text{,\ \ for }
    \vec{x} \in \Omega
\end{equation}
and
\begin{equation}\label{eq:orig-ns-bc}
    \left\{
        \begin{aligned}
            &\vec{u}(\vec{x}, t) = \vec{u}_{bc}(\vec{x}, t) \\
            &p(\vec{x}, t) = p_{bc}(\vec{x}, t)
        \end{aligned}
        \text{,\ \ for }
        \vec{x} \in \Gamma_{bc}
        \text{ and }
        t \in [0, T]
    \right.
\end{equation}
We only list the Dirichlet BCs to save space and for simplicity, as the treatment of Neumann BCs does not differ from treating PDEs in PINNs.

In the method of PINNs, the first step is to approximate the solutions of \eqref{eq:orig-ns} with a neural network model.
Let the network model name be $G$, of which the inputs are spatial coordinates $\vec{x}\equiv\begin{bsmallmatrix}x & y & z\end{bsmallmatrix}^\mathsf{T}$, time $t$, and a set of free parameters $\Theta$ that we need to determine later:
\begin{equation}\label{eq:G-network}
    \begin{bmatrix}
        \vec{u}(\vec{x}, t) \\ p(\vec{x}, t)
    \end{bmatrix}
    \approx
    G(\vec{x}, t; \Theta)
    =
    \begin{bmatrix}
        G^{\vec{u}} \\
        G^p
    \end{bmatrix}
    =
    \begin{bmatrix}
        G^u \\
        G^v \\
        G^w \\
        G^p
    \end{bmatrix}
\end{equation}
where $\vec{u}\equiv\begin{bsmallmatrix}u & v & w\end{bsmallmatrix}^\mathsf{T}$ is the velocity vector, and $p$ is the pressure.
In \eqref{eq:G-network}, we use a single network to predict both pressure and velocity fields.
Though it is possible to use separate networks for different fields, we keep \eqref{eq:G-network} this way for simplicity.
Also, later in this work, we will use $G^{\vec{u}}$, $G^u$, $G^v$, $G^w$, and $G^p$ to denote the predicted velocity vector, components, and pressure from $G$.
In this section, we will focus on the solution workflow \cite{dissanayake_neural-network-based_1994,lagaris_artificial_1998,cai_physics-informed_2021} and leave the mathematical expression of neural network models to section \ref{sec:mlp}.

The next step is to approximate the derivatives required for the differential equations.
For example, the continuity equation $\nabla \cdot \vec{u} = \pdiff{u}{x} + \pdiff{v}{y} + \pdiff{w}{z}$ requires the approximations for the 1st-order derivatives of velocity.
A possible approach is to rely on numerical differentiation.
For example, using finite difference, $\pdiff{u}{x}$ can be
\begin{equation*}
    \pdiff{u}{x} \approx \frac{G^u(x+\Delta x, \cdots)-G^u(x-\Delta x, \cdots)}{2\Delta x} 
\end{equation*}
Model $G$ is continuous and not defined on discretized space, so $x$ can be arbitrary coordinates in the domain.
And $\Delta x$ is a user-provided scalar as there is no grid.
However, we will not proceed with numerical differentiation. 
Being a continuous model and having a well-defined mathematical expression, $G$ gives us two more reasonable options.

The first option is to obtain analytical derivatives through symbolic differentiation or manual derivation.
For example, a very simple MLP network for a steady 2D flow may look like
\begin{equation}\label{eq:explicit-toy-mlp}
    \begin{aligned}
    &\begin{bmatrix}
        u &
        v &
        p
    \end{bmatrix}^\mathsf{T}
    \approx
    \begin{bmatrix}
        G^u &
        G^v &
        G^p
    \end{bmatrix}^\mathsf{T}
    =
    G(x, y; \Theta) \\
    = 
    &\begin{bmatrix}
    c_{11} \cos{\left(a_{11}x + a_{12}y + b_1\right)} + 
      c_{12} \cos{\left(a_{21}x + a_{22}y + b_2\right)} \\
    c_{21} \cos{\left(a_{11}x + a_{12}y + b_1\right)} + 
      c_{22} \cos{\left(a_{21}x + a_{22}y + b_2\right)} \\
    c_{31} \cos{\left(a_{11}x + a_{12}y + b_1\right)} + 
      c_{32} \cos{\left(a_{21}x + a_{22}y + b_2\right)} \\
    \end{bmatrix}
    \end{aligned}
\end{equation}
where $a_{ij}$, $b_i$, and $c_{ki}$ for $i$, $j=1$, $2$ and $k=1$, $2$, $3$ are free parameters in the model and belong to $\Theta$.
This MLP network is said to have one hidden layer with 2 neurons per layer and uses cosine for the activation function.
(Note that it is uncommon to use cosine for activation in real applications.)
We will formally introduce the definitions of these terms and the MLP's matrix form in section \ref{sec:mlp}.
Given the expression of \eqref{eq:explicit-toy-mlp}, we are able to obtain the derivatives.
Such as
\begin{equation}
    \begin{aligned}
    \pdiff{u}{x}
    \approx
    \pdiff{G^u}{x}
    =
    a_{11}c_{11}\cos\left(a_{11}x + a_{12}y + b_1\right) + a_{21}c_{12}\cos\left(a_{21}x + a_{22}y + b_2\right)
    \end{aligned}
\end{equation}
And the same concept applies to all higher-order derivatives.

Analytical derivatives were used in earlier literature when the networks might be just slightly more complicated than the one shown in \eqref{eq:explicit-toy-mlp}.
When the neural network models become more and more complicated, analytical derivatives become much less feasible, if not impossible.
Even modern hardware and symbolic differentiation software may not be able to obtain the analytical derivatives when $G$ has tens of thousands of free parameters and when the activation is not as simple as a cosine function.

Modern deep learning instead relies on the second option: automatic differentiation \cite{griewank_automatic_1988}.
Automatic differentiation is based on the chain rule and gives exact derivatives (exact with respect to actual computer intrinsic operations).
Section \ref{sec:ad} will talk about the fundamental concept of automatic differentiation.
At this point, we can just treat the automatic differentiation as a black box function that returns exact derivatives.
It requires fewer computing resources than symbolic differentiation and provides more accurate derivatives than numerical differentiation.

Moving on to the next step, once we obtain the desired derivatives, we substitute these derivatives into the governing PDEs, IC, and BCs.
For example, by substituting $G$ and corresponding derivatives to \eqref{eq:orig-ns}, \eqref{eq:orig-ns-ic}, and \eqref{eq:orig-ns-bc}, we get several residual functions
\begin{equation}\label{eq:residuals}
    \begin{array}{ll}
        \left\{
            \begin{array}{l}
            \hat{\vec{r}}_{m}(\vec{x}, t; \Theta) \equiv \frac{\partial G^{\vec{u}}}{\partial t}+(G^{\vec{u}} \cdot \nabla) G^{\vec{u}}+\frac{1}{\rho} \nabla G^p -\nu \nabla^{2} G^{\vec{u}} \\
            \hat{r_{c}}(\vec{x}, t; \Theta) \equiv \nabla \cdot G^{\vec{u}} 
            \end{array}
        \right. &
        \forall
        \left\{
            \begin{array}{l}
                \vec{x} \in \Omega \\ t \in [0, T]
            \end{array}
        \right. \\
        \left\{
            \begin{array}{l}
            \hat{\vec{r}}_{ic,\vec{u}}(\vec{x}; \Theta) \equiv G^{\vec{u}}(\vec{x}, t=0)-\vec{u}_0(\vec{x}) \\
            \hat{r}_{ic,p}(\vec{x}; \Theta) \equiv G^{p}(\vec{x}, t=0)-p_0(\vec{x})
            \end{array}
        \right. &
        \forall
        \vec{x} \in \Omega \\
        \left\{
            \begin{array}{l}
            \hat{\vec{r}}_{bc,\vec{u}}(\vec{x}, t; \Theta) \equiv G^{\vec{u}}-\vec{u}_{bc} \\
            \hat{r}_{bc,p}(\vec{x}, t; \Theta) \equiv G^{p}-p_{bc}
            \end{array}
        \right. &
        \forall
        \left\{
            \begin{array}{l}
                \vec{x} \in \Gamma_{bc} \\ t \in [0, T]
            \end{array}
        \right.
    \end{array}
\end{equation}
To make the expression cleaner, we omitted $\vec{x}$, $t$, and $\Theta$ in $G$.
Residual vectors $\hat{\vec{r}}_i$ and residual scalar $\hat{r}_i$, where $i$ denotes different residual sources, are functions of $\vec{x}$, $t$, and $\Theta$.
They are zero anywhere in the domain of interest only when the approximation solution $G$ perfectly fits in \eqref{eq:orig-ns}, \eqref{eq:orig-ns-ic}, and \eqref{eq:orig-ns-bc}.

The final step is to find a set of parameters $\vec{\theta}=\begin{bsmallmatrix}\theta_1 & \theta_2 & \cdots\end{bsmallmatrix}^\mathsf{T}$ that makes all residuals zero, i.e., when $\Theta=\vec{\theta}$, $G$ perfectly fits into \eqref{eq:orig-ns}, \eqref{eq:orig-ns-ic}, and \eqref{eq:orig-ns-bc}.
$\vec{\theta}$ then represents the common zero roots of all the residual functions in \eqref{eq:residuals} for all $\vec{x} \in \Omega$, $\vec{x} \in \Gamma_{bc}$, and $t\in[0$, $T]$.

Assume there are a total of $N_\Theta$ free parameters.
If residual functions in (\ref{eq:residuals}) are not complicated, and if $N_\Theta$ is small enough, we may numerically find the zero roots by solving a system of $N_\Theta$ nonlinear equations.
Such a system of nonlinear equations may be generated by enforcing the zero-residual conditions only on a set of $N_\Theta$ spatial-temporal points, i.e., $N_\Theta$ pairs of $\vec{x}$ and $t$.
However, this approach rarely results in an easy-to-solve system, given that $G$ is usually complicated and $N_\Theta$ is large.
Even for the toy model like \eqref{eq:explicit-toy-mlp}, the first-order derivatives are non-trivial.
If we further consider higher-order derivatives and the native nonlinearity in the PDEs, even this toy model results in a complicated nonlinear system. 
We do not even know if zero roots exist for this nonlinear system.

We instead relax the condition in PINNs.
We do not seek the zero roots of \eqref{eq:residuals} but just hope the desired set of parameters, $\theta$, makes the residuals sufficiently close to zero.
Also, note that the residual functions in \eqref{eq:residuals} are continuous in the domain and time range of our interest.
To make the optimization easier, we also limit ourselves to finding the minimal residuals at some spatial-temporal coordinates only, rather than asking for minimal residuals everywhere in the continuous space-time domain. 
In other words, the resulting optimization problem tries to find $\vec{\theta}$ that gives the minimal residuals on some spatial-temporal points.

Assume we have picked $N_{PDE}$ spatial-temporal points from $\vec{x}\in\Omega$ and $t\in[0$, $T]$, $N_{IC}$ points in $\Omega$ but with a fixed time $t=0$, and $N_{BC}$ points from $\vec{x} \in \Gamma_{bc}$ and $t\in[0$, $T]$.
We define the residuals at these points as \\
\begin{equation}\label{eq:residual-norms}
    \begin{aligned}
        & r_{m}(\Theta) \equiv \sum\limits_{i=1}^{N_{PDE}} \lVert\frac{\partial G_i^{\vec{u}}}{\partial t}+(G_i^{\vec{u}} \cdot \nabla) G_i^{\vec{u}}+\frac{1}{\rho} \nabla G_i^p -\nu \nabla^{2} G_i^{\vec{u}} \rVert^2 \\
        & r_{c}(\Theta) \equiv \sum\limits_{i=1}^{N_{PDE}} ( \nabla \cdot G_i^{\vec{u}} )^2 \\
        & r_{ic,\vec{u}}(\Theta) \equiv \sum\limits_{i=1}^{N_{IC}} \lVert G^{\vec{u}}(\vec{x}_i, t=0)-\vec{u}_0(\vec{x}_i) \rVert^2 \\
        & r_{ic,p}(\Theta) \equiv \sum\limits_{i=1}^{N_{IC}} ( G^{p}(\vec{x}_i, t=0)-p_0(\vec{x}_i) )^2 \\
        & r_{bc,\vec{u}}(\Theta) \equiv \sum\limits_{i=1}^{N_{BC}} \lVert G_i^{\vec{u}}-\vec{u}_{bc} \rVert^2 \\
        & r_{bc,p}(\Theta) \equiv \sum\limits_{i=1}^{N_{BC}} ( G_i^{p}-p_{bc} )^2
   \end{aligned}
\end{equation}
where $\begin{bsmallmatrix}G_i^{\vec{u}} & G_i^p\end{bsmallmatrix}^\mathsf{T} = G(\vec{x}_i, t_i; \Theta)$, and $\lVert\cdot\rVert$ denotes $l_2$ norms.

One approach to interpret the optimization problem is
\begin{equation}\label{eq:hard-constraint-loss}
    \begin{aligned}
    &\vec{\theta} = \operatorname*{arg\,min}\limits_{\Theta} \left[r_m(\Theta) + r_c(\Theta)\right] \\
    \text{ subject to } &r_{ic,\vec{u}}=r_{ic,p}=r_{bc,\vec{u}}=r_{bc,p}=0
    \end{aligned}
\end{equation}
This is a constrained optimization problem.
Only residuals from PDEs are relaxed, and residuals from BCs and IC must be exactly zero.
Though exactly satisfied BCs and IC sound attractive, optimization with hard constraints is much more difficult than unconstrained optimization.
And optimization methods may not be general enough for arbitrary types of BCs and arbitrary computational domains.
This approach was used in, for example, references \cite{lagaris_artificial_1998,McFall2009,mcfall_solving_2010,berg_unified_2018} for limited types of PDEs and applications.

Alternatively, most recent reports of PINNs used soft constraints, that is
\begin{equation}\label{eq:total-residual}
    \vec{\theta} = \operatorname*{arg\,min}\limits_{\Theta} \left[
        r_m(\Theta) + r_c(\Theta) + r_{ic,\vec{u}}(\Theta) + r_{ic,p}(\Theta) + r_{bc,\vec{u}}(\Theta) + r_{bc,p}(\Theta)
    \right]
\end{equation}
Though the optimized $\vec{\theta}$ does not guarantee that IC and BCs are satisfied, \eqref{eq:total-residual} is easier to solve and to be generalized to different PDEs, BCs, or applications.
In the terminology of optimization, \eqref{eq:total-residual} is called the loss function or objective.
Residuals from different sources are called loss terms.

In practice, a more general form of \eqref{eq:total-residual} is
\begin{equation}\label{eq:total-residual-weighted}
    \begin{aligned}
        \vec{\theta}
        &=
        \operatorname*{arg\,min}\limits_{\Theta} r(\Theta)  \\
        &=
        \operatorname*{arg\,min}\limits_{\Theta} \sum \alpha_i r_i(\Theta)
    \end{aligned}
\end{equation}
where $i$ denotes different loss terms in \eqref{eq:residual-norms}.
Each loss term is weighted before being aggregated to the final loss.
To our best knowledge, there is not yet a standard or guideline for how to properly configure $\alpha_i$.
It is common to see $\alpha_i=1$ in the literature.

If there exist zero roots that make all residuals in \eqref{eq:residual-norms} zero, then a perfect optimization method will find them from either \eqref{eq:hard-constraint-loss}, \eqref{eq:total-residual}, or \eqref{eq:total-residual-weighted}.
This hope is surely unrealistic.
First, zero roots' existence is not guaranteed.
Second, the hypersurface of the aggregated loss, $r(\Theta)$, in the vector space of $\Theta$ is usually neither convex nor concave, making it difficult to find the global minimum.
Even a simple 1D linear Burger's equation with an MLP of around \num{7850} free model parameters shows a complicated hypersurface of $r(\Theta)$ \cite{krishnapriyan_characterizing_2021}.
While finding zero roots or the global minimum on the hypersurface of $r(\Theta)$ is difficult, practitioners of PINNs just assume the local minimums of $r(\Theta)$ are good and small enough.

This poses a fundamental difference between conventional CFD schemes and PINNs: the former seeks exact zero roots to make residuals zero, while the latter just {\it hopes} to make residuals as small as possible.
It is thus reasonable to doubt PINNs' accuracy compared to conventional schemes.

Nevertheless, finding optimized $\vec{\theta}$ concludes the workflow of data-free PINNs for solving PDEs.
Figure \ref{fig:pinn-workflow} shows a graphical illustration of the workflow.
\begin{figure}[hbt!]
    \includegraphics[width=\linewidth]{figs/pinn.tikz} 
    \caption{A graphical demonstration of workflow in PINNs}
    \label{fig:pinn-workflow}
\end{figure}

The optimization process is done with the Adam optimizer \cite{kingma_adam_2017} in this work, together with a batched approach described in the next section.
% vim:ft=tex


    \subsection{Batched Training}\label{sec:batched-training}
    %! TEX root = main.tex

Compared to solving $N_\Theta$ nonlinear equations directly, an optimization problem of \eqref{eq:total-residual} or \eqref{eq:total-residual-weighted} allows us to use any number of spatial-temporal points.
That is, there is no limitation on the magnitude of $N_{PDE}$, $N_{BC}$, and $N_{IC}$.
This eases the need for computational resources.
In data-free PINNs, these spatial-temporal points are called training points or training data.
They are spatial-temporal coordinates where we evaluate the residuals of PDEs, IC, and BCs.
While training points may be generated manually and intentionally positioned at desired locations (just like meshing in conventional CFD simulations), it is more common to generate them by selecting coordinates randomly using a uniform density function. 

The ability to use arbitrary numbers of training points helps the generalizability of a trained PINNs model.
When more training points are involved in \eqref{eq:residual-norms} and hence \eqref{eq:total-residual-weighted}, the minimal residuals are achieved at more locations in the domain. 
Ultimately, with unlimited training points evenly distributed in the domain, optimizing the discretized residuals in \eqref{eq:residual-norms} becomes optimizing continuous residual functions in \eqref{eq:residuals}. 
And the resulting optimal parameters, $\vec{\theta}$, will make $G$ a more accurate approximation solution to the whole domain of $\Omega$ and $t\in[0$, $T]$.

More points also mean a higher computational cost.
Batched training is thus utilized to reduce the cost.
For example, if using an iterative optimization method (such as the gradient-descent or Adam optimizer), we can always generate a new batch of training points for a new iteration to evaluate the discretized residuals.
After a significant number of iterations, the total number of training points involved in optimization will also be significant.
Once the model is trained, theoretically, it is capable of giving accurate predictions at any location and time.

In practice, however, it is not a cheap task if thousands or even more random points have to be generated on the fly at each iteration.
We instead generated a fixed amount of training points before the optimization and only used a batch of them in each iteration.
For example, to use $N_{PDE}$ points to evaluate the PDE residuals at each iteration, we may generate $N_{PDE}\times 1000$ points in advance and divide them into $1000$ batches.
If we run an optimizer for 1 million iterations, each batch is repeated every 1000 iterations.

Theoretically, each batch should have similar statistical properties and give a similar gradient vector under fixed model parameters.
That is
\begin{equation}
\left.\nabla_\theta r(\Theta) \right|_{\vec{x}, t \in \mathcal{X}_1} \sim
\left.\nabla_\theta r(\Theta) \right|_{\vec{x}, t \in \mathcal{X}_2} \sim
\cdots
\end{equation}
where $\mathcal{X}_i$ for $i=1$, $2$, $\cdots$ represents the $i$-th batch of the training points.
These gradients are used to update $\vec{\theta}$ in iterative optimization methods.
In other words, the hypersurface of $r(\Theta)$ is expected not to change significantly from iteration to iteration when using different batches of points to evaluate it.
However, whether this statement is true may depend on the number of points in each batch: the number of points should be large enough to have similar statistical properties across different batches.

In PINNs, as the training points are randomly sampled from the computational space-time domain, the number of points in a batch should be large enough to cover all over the domain.
Otherwise, for example, if one batch mostly covers the inflow region of a cylinder flow while the next batch covers mostly the wake region behind the cylinder, then the hypersurface of the PDE residuals will change significantly.
This change of the hypersurface from iteration to iteration may slow down the convergence.
In the later benchmarks, we would like to examine the effect of batch sizes, i.e., the number of points in each batch. 

% vim:ft=tex


    \subsection{Adaptive Loss Aggregation}\label{sec:loss-annealing}
    %! TEX root = main.tex
In equation \eqref{eq:total-residual-weighted}, each individual loss term is weighted.
However, how to properly assign the weights is still and open question.
In references \cite{jin_nsfnets_2020,wang_understanding_2021}, the authors proposed an annealing approach to change these weights in an adaptive fashion during the optimization process.
However, these adaptive approaches have not been further tested by more works.
In this work, part of the benchmarks will cover the performance of the adaptive weight strategy proposed in \cite{jin_nsfnets_2020}.

Jin et al. \cite{jin_nsfnets_2020} proposed the following annealing loss aggregation algorithm for iterative optimization methods:
\begin{equation}
    r^k(\Theta) = r_{PDE}^k(\Theta) + 
        \left(\left(1-\lambda\right)\alpha^k + \lambda\alpha^{k+1}\right)r_{BC}^k(\Theta) + 
        \left(\left(1-\lambda\right)\beta^k + \lambda\beta^{k+1}\right)r_{IC}^k(\Theta)
\end{equation}
where $k$ denotes the $k$-th iteration in an iterative optimization method.
The subscript $PDE$, $BC$, and $IC$ denote the loss contributions from the residuals of PDEs, BCs, and IC.
$\lambda$ is a user-provided parameter to control the moving average of the current and the previous weights.
The concept of this adaptive approach is to make the gradients of each loss term comparable.
When using the gradient-descent method and its derived methods, the magnitude of each loss term is reduced with a similar rate.
And
\begin{equation}
    \alpha^{k+1} = \frac{\overline{\lvert\nabla_\Theta r_{PDE}^k(\Theta)\rvert}}{\overline{\lvert\nabla_\Theta r_{BC}^k(\Theta)\rvert}}
    \text{\ \ \ \ and\ \ \ \ }
    \beta^{k+1} = \frac{\overline{\lvert\nabla_\Theta r_{PDE}^k(\Theta)\rvert}}{\overline{\lvert\nabla_\Theta r_{IC}^k(\Theta)\rvert}}
\end{equation}
$\lvert\cdot\rvert$ denotes the element-wise absolute values of a vector.
$\overline{\lvert\cdot\rvert}$ is the mean values of these absolute values.

The following expression better represents the actual implementation in our code:
\begin{equation}
    \begin{aligned}
        &\zeta^k =
            \overline{\lvert\nabla_\Theta r_{m,u}^k(\Theta)\rvert} +
            \overline{\lvert\nabla_\Theta r_{m,v}^k(\Theta)\rvert} +
            \overline{\lvert\nabla_\Theta r_{m,w}^k(\Theta)\rvert} +
            \overline{\lvert\nabla_\Theta r_{c}^k(\Theta)\rvert} \\
        &\alpha^{k+1} = 
            \frac{\zeta^k}{\overline{\lvert\nabla_\Theta r_i^k(\Theta)\rvert}} \\
        &r^k(\Theta) = r_m^k(\Theta) + r_c^k(\Theta) + 
            \sum\limits_{i} \left(\left(1-\lambda\right)\alpha_i^k + \lambda\alpha_i^{k+1}\right) r_i^k(\Theta)
    \end{aligned}
\end{equation}
where $r_{m,u}^k$, $r_{m,v}^k$, and $r_{m,w}^k$ are the $u$-, $v$-, and $w$-component in the residual vector of the momentum equations at $k$-th iteration.
And subscript $i$ denotes different loss terms, excluding $r_m(\Theta)^k$ and $r_c(\Theta)^k$.
In this work, $\lambda$ is fixed at $0.1$ for all cases using annealing loss aggregation.
% vim:ft=tex

\section{Deep Neural Network Modeling}\label{sec:pinn-dnnm}

    \subsection{Multilayer Perceptron Networks}\label{sec:mlp}
    %! TEX root = main.tex

In the previous section, $G$ denotes a neural network model (or any mathematical model) that predicts flow quantities at given spatial-temporal coordinates.
In this section, we would like to introduce the mathematical expression of the most common neural network model in data-free PINNs: MLP (multilayer perceptron) networks.

The universal approximation theorem \cite{hornik_approximation_1991} states that large-enough MLP networks can approximate any smooth function.
This theorem justifies the use of MLP networks as PDEs' approximate solutions.
These PDEs include the Navier-Stokes equations, which are the primary governing equations for CFD.\footnote[0]{Though the existence of the smooth solutions for the Navier-Stokes equations is not yet proved, as CFD practitioners, we usually ignore this theoretical fact.}

Figure \ref{fig:mlp-graph} is a commonly seen graphical illustration of how MLP works.
\begin{figure}[hbt!]
    \singlespacing
    \includegraphics[width=0.95\linewidth]{figs/mlp.tikz}
    \caption{Graphical illustration of MLP networks}
    \label{fig:mlp-graph}
\end{figure}
To avoid introducing more mathematical symbols, in this section and figure \ref{fig:mlp-graph}, we reuse the notation from the previous section.

An MLP network is fundamentally a series of linear-nonlinear mapping pairs:
\begin{equation}\label{eq:mlp-formula}
    \begin{array}{ll}
        \vec{h}^0 \equiv \begin{bmatrix} \vec{x} \\ t \end{bmatrix} & \\
        \vec{h}^k = \sigma_{k-1}\left(A^{k-1}\vec{h}^{k-1}+\vec{b}^{k-1}\right)\text{,} & 1 \le k \le N_l \\
        \vec{h}^{N_l+1}\equiv \begin{bmatrix} G^{\vec{u}} \\ G^p \end{bmatrix} = \sigma_{N_l}\left(A^{N_l}\vec{h}^{N_l}+\vec{b}^{N_l}\right) &
    \end{array}
\end{equation}
$\vec{h}^k$ is called a hidden layer or a latent space vector.
$\sigma_{k}$ is a chosen element-wise nonlinear function.
$A^{k}$ and $\vec{b}^k$ for $k=0$ to $k=N_l$ are parameter matrices and vectors that consist of free model parameters.
In other words, $\Theta=\left\{A^0, \vec{b}^0, A^1, \vec{b}^1, \cdots \right\}$.
$N_l$ denotes the number of hidden layers, and $N_n$ in figure \ref{fig:mlp-graph} represents the number of elements in each hidden layer, $\vec{h}^k$.
Elements in $\vec{h}^k$ are often called neurons in deep learning.
Theoretically, $N_n$ does not have to be constant.
Each $\vec{h}^k$ can have different $N_n$.
However, as there is no clear guideline on specifying variate $N_n$ for PINNs, most reports in PINNs just use a constant $N_n$ across hidden layers.
We follow the same approach in this work.

$A^k$ and $\vec{b}^k$ are the free parameters we want to determine in the optimization step (i.e., equation \eqref{eq:total-residual-weighted}).
$N_l$ and $N_n$ are also parameters that control the complexity and accuracy of an MLP network.
However, they are determined by users rather than by optimization.
These parameters are called hyperparameters---they are not counted in the degrees of freedom of a mathematical model.

The element-wise nonlinear functions $\sigma_{k}$ are also pre-determined by users.
Theoretically, they can be different functions across different $k$.
To our knowledge, most reports in PINNs use the same function for all $k$, i.e., $\sigma_0(x)=\sigma_1(x)=\cdots=\sigma(x)$.
In PINNs, the most commonly seen choices of $\sigma$ are hyperbolic tangent ($\tanh$) and the sigmoid function:
\begin{equation}
    \operatorname{sigmoid}(x) = \frac{1}{1+\exp(-x)}
\end{equation}
In this work, we use SiLU for $\sigma$ \cite{hendrycks_gaussian_2020}, which is the default option for Modulus:
\begin{equation}\label{eq:silu}
    \operatorname{SiLU}(x) = \frac{x}{1+\exp(-x)}
\end{equation}
To our best knowledge, in terms of solving flow problems with data-free PINNs, no reports have discussed the effect of $\sigma$ except for reference \cite{li_integration_2010}.

While simple MLP networks like \eqref{eq:mlp-formula} are not popular in modern deep learning applications, they are still the first choice in solving PDEs with data-free PINNs.
The reason may be that the universal approximation theorem proves MLP networks' ability to approximate any smooth function, which makes them theory-backed mathematical models for PDEs' general solutions.
Other types of networks lack such a theoretical backup.
Nevertheless, lacking theoretical proof does not mean other networks would not work.
For example, reference \cite{sirignano_dgm:_2018} reports the success of using an LSTM-style network to solve PDEs.

Networks with higher complexity imply a higher computational cost.
Whether using complicated networks is worth it may be a question from the perspective of cost-performance ratios.
Instead, in this work, we used a variant of MLP networks implemented in Modulus---a weight-normalized MLP \cite{salimans_weight_2016}:
\begin{equation}\label{eq:weighted-norm-mlp-formula}
    \begin{array}{ll}
        \vec{h}^0 \equiv \begin{bmatrix} \vec{x} \\ t \end{bmatrix} & \\
        \vec{h}^k = \sigma_{k-1}\left(
            \operatorname{diag}\left(\vec{w}_g^{k-1}\right)
            \operatorname{diag}\left(\vec{w}_n^{k-1}\right)^{-1}
            W^{k-1}\vec{h}^{k-1}+\vec{b}^{k-1}
        \right)\text{,} & 1 \le k \le N_l \\
        \vec{h}^{N_l+1}\equiv \begin{bmatrix} G^{\vec{u}} \\ G^p \end{bmatrix} = \sigma_{N_l}\left(A^{N_l}\vec{h}^{N_l}+\vec{b}^{N_l}\right) &
    \end{array}
\end{equation}
$\vec{w}_g^{k}$ for $k=0,\cdots,N_l$ represents a parameter vector.
$W^k$ is a parameter matrix.
$\vec{w}_n^{k}$ is not a parameter vector but holds the norms of each row in $W^k$.
The key idea is to decouple each row in $A^k$ to a length multiplying a direction.
The matrix $\operatorname{diag}\left(\vec{w}_n^k\right)^{-1}W^k$ is row-normalized, in which each row represents a direction.
And each element in $\vec{w}_g^k$ represents the lengths of the corresponding direction.
If the matrix $A^k$ is given, then $\vec{w}_g^k$ is the length of each row in $A^k$.
In other words, $\operatorname{diag}\left(\vec{w}_g^{k}\right) \operatorname{diag}\left(\vec{w}_n^{k}\right)^{-1} W^{k}$ equals to $A^k$.
When $A^k$ is unknown and needs to be trained, this decoupling allows the lengths and directions to be trained separately.

Compared to using other types of networks, a weight-normalized MLP \eqref{eq:weighted-norm-mlp-formula} does not actually alter the architecture of an MLP network.
It just rearranges the mathematical expression.
Theoretically, either using \eqref{eq:mlp-formula} or \eqref{eq:weighted-norm-mlp-formula}, a perfect optimization scheme should find $A^k$ and $\operatorname{diag}\left(\vec{w}_g^{k}\right) \operatorname{diag}\left(\vec{w}_n^{k}\right)^{-1} W^{k}$ equivalent to each other.
However, the real world is far from perfect.
Salimans and Kingma \cite{salimans_weight_2016} found that using the expression of \eqref{eq:weighted-norm-mlp-formula} speeds up the convergence when using 1st-order optimization schemes, such as the basic gradient-descent and Adam.

The weight-normalized variant of the MLP networks has the following formula to calculate the total number of free parameters:
\begin{equation}\label{eq:dof-calculator}
    DoF =
     N_{n} \left(N_{in} + 2\right) + 
    N_{n} \left(N_{n} + 2 \right) \left(N_l-1\right) +
     N_{out}\left(N_{n} + 1\right)
\end{equation}
where $DoF$ means the degrees of freedom, another term for the number of free parameters.
% vim:ft=tex


    \subsection{Handling of Periodic Boundaries}\label{sec:periodic-boundary}
    %! TEX root = main.tex

When dealing with periodic boundary conditions, to calculate the loss (i.e., $\vec{r}_{bc,\vec{u}}$ and $\vec{r}_{bc, p}$), one possible approach is to define the loss terms as the difference in the values between the pair of the corresponding boundaries.
For example, assume we have a periodic BC on $x=0$ and $x=L$.
We define
\begin{equation}\label{eq:naive-periodic-bc}
    \begin{aligned}
    &r_{bc,\vec{u}}(\Theta) = \sum\limits_{i=1}^{N_{BC}} \lVert G^{\vec{u}}(\vec{x}_{0,i}, t_i; \Theta) - G^{\vec{u}}(\vec{x}_{L,i}, t_i; \Theta) \rVert^2\\
    &r_{bc,p}(\Theta) = \sum\limits_{i=1}^{N_{BC}} ( G^{p}(\vec{x}_{0,i}, t_i; \Theta) - G^{p}(\vec{x}_{L,i}, t_i; \Theta) )^2
    \end{aligned}
\end{equation}
where $\vec{x}_{0,i}$ and $\vec{x}_{L,i}$ are spatial coordinates on the plane of $x=0$ and $x=L$, respectively.
And coordinate components of $y$ and $z$ in $\vec{x}_{0,i}$ and $\vec{x}_{L,i}$ should match.

Modulus provides an alternative approach to handle periodic BCs.
Consider a pair of periodic BCs on $x=x_1$ and $x=x_2$.
We notice that $\sin(2\pi\frac{x-x_1}{x_2-x_1})$ and  $\cos(2\pi\frac{x-x_1}{x_2-x_1})$ have a period of $x_2-x_1$.
Hence, we expand the inputs of $G$ to
\begin{equation}\label{eq:periodic-G}
    G = G(\sin(2\pi\frac{x-x_1}{x_2-x_1}), \cos(2\pi\frac{x-x_1}{x_2-x_1}), y, z, t; \Theta)
\end{equation}
If component $y$ or $z$ also has periodic BCs, they are converted as well following the same logic.
In the present work, the periodic BCs are handled using this approach.
% vim:ft=tex

    \subsection{Automatic Differentiation}\label{sec:ad}
    %! TEX root = main.tex
The PINN method relies on automatic differentiation to evaluate the derivatives of $G$.
Automatic differentiation algorithms record how a numeric value is calculated and then apply the chain rule to determine that value's derivatives with respect to its inputs.
For example, assume a piece of code evaluates the following calculation:
\begin{equation}\label{eq:graph-example}
 y = \left.{e^{\left(x_1x_2-\sin{\left(x_2\right)}\right)}+3x_1}\right|_{\begin{subarray}{l}x_1=1.2 \\ x_2=0.5\end{subarray}} = 4.72\dots
\end{equation}
On computers, the calculation is realized by a sequence of basic unary and binary operations on numeric values.
Automatic differentiation algorithms record such a sequence in a computational graph, as illustrated in figure~\ref{fig:automatic-differentiation-forward}.
\begin{figure}[hbt!]
    \Centering
    \begin{minipage}[c]{0.4\textwidth}
        \includegraphics[width=\linewidth]{figs/automatic-differentiation-forward.tikz}
    \end{minipage}%
    \begin{minipage}[c]{0.5\textwidth}
        \scriptsize
        \singlespacing
        \begin{equation*}
            \begin{aligned}
                &v_0 = \mathrm{assign}\left(1.2\right) \\
                &v_1 = \mathrm{assign}\left(0.5\right) \\
                &v_2 = \mathrm{assign}\left(3\right) \\
                &v_3 = \mathrm{multiply}\left(v_2, v_0\right)  = 3 \times 1.2 = 3.6 \\
                &v_4 = \mathrm{multiply}\left(v_0, v_1\right) = 1.2 \times 0.5 = 0.6 \\
                &v_5 = \sin\left({v_1}\right) = \sin\left(0.5\right) = 0.479\dots \\
                &v_6 = \mathrm{negate}\left(v_5\right) = -0.479\dots \\
                &v_7 = \mathrm{add}\left(v_4, v_6\right) = 0.12\dots \\
                &v_8 = \exp\left({v_7}\right) = 1.128\dots \\
                &v_9 = \mathrm{add}\left(v_3, v_8\right) = 4.728\dots
            \end{aligned}
        \end{equation*}
    \end{minipage}
    \caption{Computational graph of equation~\ref{eq:graph-example}}%
    \label{fig:automatic-differentiation-forward}
\end{figure}

It is relatively easy to evaluate the derivatives of unary and binary functions with respect to their inputs.
Combining this fact and the chain rule, automatic differentiation evaluates:
\begin{equation}\label{eq:reverse-graph-example}
    \left\{
        \begin{array}{ll}
            d_9 = \pdiff{v_9}{v_9} = 1 & \\
            d_i = \pdiff{v_9}{v_i}
                = \sum\limits_{p\in P(i)}\pdiff{v_9}{v_p}\pdiff{v_p}{v_i}
                = \sum\limits_{p\in P(i)}d_p\pdiff{v_p}{v_i}, & \text{for }i=8, 7, \ldots, 1, 0
        \end{array}
    \right.
\end{equation}
The set $P(i)$ denotes the parents of node $i$ in the reversed computational graph.
For example, as shown in figure~\ref{fig:automatic-differentiation-backward}, the parents of node $1$ are $4$ and $5$.
Any derivatives with respect to $v_1$ have contributions from nodes $4$ and $5$.
Note that we evaluate equation~\ref{eq:reverse-graph-example} in a reversed order, $i=9$, $8$, $7$, $\cdots$, so that $d_p$ ($\forall p\in P(i)$) are ready when evaluating any $d_i$.
See figure~\ref{fig:automatic-differentiation-backward} for the backward calculation.
\begin{figure}[hbt!]
    \begin{minipage}{0.4\textwidth}
        \includegraphics[width=\linewidth]{figs/automatic-differentiation-backward.tikz}
    \end{minipage}%
    \begin{minipage}{0.5\textwidth}
        \scriptsize
        \singlespacing
        \begin{equation*}
            \begin{aligned}
                d_9& = \pdiff{v_9}{v_9} = 1 & d_8& = d_9 \pdiff{v_9}{v_8} = 1 \\[-2pt]
                d_7& = d_8 \pdiff{v_8}{v_7} = 1.128\dots & d_6& = d_7 \pdiff{v_7}{v_6} = 1.128\dots \\[-2pt]
                d_5& = d_6 \pdiff{v_6}{v_5} = -1.128\dots & d_4& = d_7 \pdiff{v_7}{v_4} = 1.128\dots \\[-2pt]
                d_3& = d_9 \pdiff{v_9}{v_3} = 1 & d_2& = d_3 \pdiff{v_3}{v_2} = 1.2
            \end{aligned}
        \end{equation*}%
        \begin{equation*}
            \begin{aligned}
                d_1& = d_4 \pdiff{v_4}{v_1} + d_5 \pdiff{v_5}{v_1} = d_4v_0 + d_5\cos(v_1) = 0.36\dots = \pdiff{y}{x_2} \\
                d_0& = d_4 \pdiff{v_4}{v_0} + d_3 \pdiff{v_3}{v_0} = d_4v_1+d_3v_2 = 3.56\dots = \pdiff{y}{x_1}
            \end{aligned}
        \end{equation*}
    \end{minipage}
    \caption{Computational graph of automatic differentiation of equation~\ref{eq:graph-example}}%
    \label{fig:automatic-differentiation-backward}
\end{figure}

Given that $v_9=y$, $v_0=x_1$ and $v_1=x_2$, once the backward calculation reaches $i=0$ and $i=1$, we automatically obtain $\pdiff{y}{x_1}$ and $\pdiff{y}{x_2}$.
Please refer to reference~\cite{griewank_automatic_1988} for a detailed introduction to automatic differentiation algorithms.

No matter how complicated a mathematical expression is (like a neural network $G$), it is always broken down to a sequence of simple binary or unary operations.
And hence automatic differentiation guarantees the exact derivatives in terms of this sequence of unary and binary computation.

Automatic differentiation is used to evaluate the derivatives of $G$'s outputs with respect to its inputs.
Also, it is used to evaluate the derivatives of the residuals with respect to model parameters, i.e., $\nabla_{\Theta} r(\Theta)$.
The latter is used by optimizers in the optimization process.
% vim:ft=tex


\section{Other Training Strategies}
%! TEX root = main.tex
The later benchmarks of PINNs in this work include some training strategies that have not been reported in the literature on PINNs.
These strategies represent some latest development in other deep learning applications beyond PINNs.
We would like to see if these strategies help improve the accuracy or reduce the computational cost of data-free PINNs.
% vim:ft=tex

    \subsection{Cyclical Learning Rates and Stochastic Weight Averaging}\label{sec:cyclic-swa}
    %! TEX root = main.tex
In iterative optimization methods that are derived from the gradient -descent method, the last step in each iteration is to update parameters with the corrected or uncorrected gradients of the loss with respect to model parameters:
\begin{equation}
    \vec{\theta}^{k+1} = \vec{\theta}^{k} - \eta\Phi(\nabla_{\Theta}r(\Theta=\vec{\theta}^k))
\end{equation}
where $\eta$ is a hyperparameter controlling the step size to move in the negative gradient direction on the hypersurface of $r(\Theta)$ in $\Theta$ space.
$\Phi(\nabla_{\Theta}r(\vec{\theta}^k))$ represents either a corrected or uncorrected gradient.
For example, $\Phi(\nabla_{\Theta}r(\vec{\theta}^k))=\nabla_{\Theta}r(\vec{\theta}^k)$ in the vanilla gradient-descent method.
And $\Phi(\nabla_{\Theta}r(\vec{\theta}^k)) = \left(1-\beta\right)\nabla_{\Theta}r(\vec{\theta}^k) + \beta \vec{g}^{k-1}$ for the gradient-descent with momentum, where $\vec{g}^{k-1}$ is the running average of $\nabla_{\Theta}r(\Theta)$ up to $k-1$-th iteration.
See reference~\cite[Section~8.3]{goodfellow_deep_2016} for the details of these optimization methods.
As the iteration approaches a local minimum, the gradient approaches zero, and eventually, the training converges to $\vec{\theta}$ when $\vec{\theta}^k=\vec{\theta}^{k-1}$

The step size $\eta$ is called {\it learning rate} in modern machine learning.
The configuration of the learning rate has always been an open question.
In earlier iterations, a large learning rate helps speed up the convergence.
However, when the training approaches a local minimum in later iterations, a large learning rate may make the step too big and miss the minimum.
So common practice is to have smaller and smaller learning rates for later iterations.
A way to adjust the learning rate without human intervention is to use a learning rate schedule based on the iteration counter.
For example, an exponential learning rate has a formula of
\begin{equation}
    \eta(k) = \eta_0 \times \gamma^k
\end{equation}
where $k$ denotes the $k$-th iteration, and $\gamma$ is a hyperparameter controlling the decaying rate.

Though a large learning rate may miss a local minimum, it sometimes also helps escape from a local minimum or a saddle point on the hypersurface of $r(\Theta)$.
With exponential learning rate adjustment, later iterations may lose the ability to find a new and better local minimum or lose the ability to quickly go over the plateaus surrounding saddle points.

Smith \cite{smith_cyclical_2017} proposed a cyclical learning rate adjusting strategy for general deep learning applications.
He found that the cyclical learning rate increased a trained model's accuracy and accelerated the training convergence.
In the cyclical learning rate, the learning rate bounces back and forth between an upper and lower bound.
The upper bound can a variable that depends on the iteration counter.

The intuitive reason why the cyclical learning rate works is that the difficulty of training deep neural networks is dominated by saddle points rather than poor local minimums \cite{dauphin_identifying_2014}.
The cyclical learning rate helps escape saddle points in later iterations by providing large learning rates at the upper bound.
And once escaping from the saddle points and in a convex region of a minimum, the small learning rates help lock the minimum.
The large learning rate is not supposed to have a strong effect after reaching a convex region and locking a minimum, as the gradients around a minimum are usually small.

In this work, we would like to try an exponential-range cyclical learning rate, also proposed in reference~\cite{smith_cyclical_2017}.
Given a half-cycle size $N_c$ (that is, it takes $N_c$ iterations for the rate to go from the lower to the upper bound or from the upper to the lower bound), a counter $c$ indicating the current cycle, and $k$ indicating the current iteration, we have
\begin{equation}\label{eq:cyclical-learning-rate}
    \eta(k) = \eta_{low} + \max(0, 1-s)\times(\eta_{high}-\eta_{low})\times\gamma^k
\end{equation}
where $s \equiv \left\lvert \frac{k}{N_c} - 2c + 1\right\rvert$ and $0 \le s \le 1$.
Figure \ref{fig:cyclic-swa-tests-lr-hist} in chapter \ref{chap:pinn-cases} shows an example of how the learning rate changes with iterations. 

Another training technique we would like to test together with the cyclical learning rate is the stochastic weight averaging (SWA) \cite{izmailov_averaging_2019}.
SWA basically averages the $\vec{\theta}^k$ from the last few iterations to get the final model parameters:
\begin{equation}
    \vec{\theta} = \frac{1}{N_{SWA}}\sum\nolimits_{i=N_k-N_{SWA}+1}^{N_k} \vec{\theta}^i
\end{equation}
where $N_{SWA}$ denotes the number of iterations to be used in the averaging, and $N_k$ represents the total iterations of the iterative optimization methods.

Izmailov et al. \cite{izmailov_averaging_2019} showed that SWA improved the trained models' accuracy and generalizability.
The intuitive mechanism behind SWA is that, on the hypersurface of $r(\Theta)$, the optimizer rarely finds an exact minimum but usually circulates in the vicinity of that location.
The last few iterations at the end of training are composed of the coordinates surrounding the actual coordinates of the minimum loss on the hypersurface of $r(\Theta)$ in the $\Theta$ vector space.
Hence, an average of the $\vec{\theta}^k$ may give us the $\vec{\theta}$ that sits at the center of these $\vec{\theta}^k$.

Another interpretation of SWA is by combining the hypersurfaces from different iterations.
Due to the batched training, the hypersurface of $r(\Theta)$ at each iteration is slightly different.
Hence, the $\vec{\theta}^k$ corresponding to the minimum at one iteration does not give the minimal loss and close-to-zero gradients at the next iteration.
SWA helps create a hypersurface with a superposition of hypersurfaces at different iterations, and this new hypersurface has a flatter region surrounding the minimum.
This wide and flat region of minimum represents the generalizability of a trained model: how well the model can predict against inputs it never or rarely saw.
And an average of $\vec{\theta}^k$ sits at the center of this wide and flat region of minimum.

As Izmailov et al. proposed using SWA together with the cyclical learning rate, we would like to include SWA in our benchmarks.


    \subsection{Nonlinear Conjugate-Gradient Optimizer and Line Search}\label{sec:ncg}
    %! TEX root = main.tex

The gradient-descent method and its derived methods all suffer from the problems of choosing the right learning rates.
The learning rate schedules adjust the value according to pre-defined formulae rather than in an adaptive fashion.
Hence, using a schedule does not resolve the problem.

Moreover, using negative gradients as the moving directions to find the minimal loss does not mean the overall descent is the fastest.
For example, at $k$-th iteration, the negative gradient $- \nabla_{\Theta} r(\vec{\theta}^k)$ just means the direction that descends the fastest at $\Theta=\vec{\theta}^k$.
It does not mean the minimum sits in this direction.
Another real-world example is reaching the lowest-elevation point in a mountain area.
Sometimes it is faster to reach the lowest point by climbing and crossing a mountain rather than always following the downhill directions.

Nonlinear conjugate-gradient (CG) methods provide an alternative searching direction, and a proper line-search algorithm helps determine the learning rate.
In this work, we implemented a variant of CG proposed by Hager and Zhang \cite{hager_new_2005,hager_survey_2006,hager_algorithm_2006}.
We also implemented an inexact line-search algorithm proposed by the same authors.
At the $k$-th iteration in CG, the general update formula is
\begin{equation}\label{eq:cg-update-formula}
    \begin{aligned}
        &\vec{d}^k = 
        \left\{
            \begin{array}{ll}
                -\vec{g}^0 & \text{if\ }k = 0 \\
                -\vec{g}^k + \beta_{k-1}\vec{d}^{k-1} & \text{otherwise}
            \end{array}
        \right. \\
        &\vec{\theta}^{k+1} = \vec{\theta}^k + \alpha_k \vec{d}^k
    \end{aligned}
\end{equation}
where $\vec{d}^k$ is the searching direction, and $\alpha_k$ is the step size (or the learning rate).
$\vec{g}^k$ is the gradient at the $k$-th iteration.
In our work, this corresponds to $\nabla_{\Theta} r(\Theta=\vec{\theta}^k)$.
The vanilla gradient-descent method can be deemed as a special case of \eqref{eq:cg-update-formula}, in which $\beta_0=\beta_1=\cdots=0$.
Generally speaking, each iteration in CG methods can be seen as starting a new search in a direction between the current search direction and the fastest descending direction at this location.

Different CG methods provide different approaches to evaluating $\beta_k$.
That is, the resulting searching directions from different CG methods are different.
In our work, we used the formula proposed in references~\cite{hager_new_2005,hager_algorithm_2006}:
\begin{equation}\label{eq:hager-cg-beta}
    \beta_k = \max\left(\hat{\beta}_k, \eta_k\right)
\end{equation}
where
\begin{equation}\label{eq:hager-cg-etak}
    \eta_k = \frac{-1}{\lVert\vec{d}^k\rVert \min \left(\eta, \lVert\vec{g}^k\rVert\right)}
\end{equation}
and
\begin{equation}\label{eq:hager-cg-beta-hat-k}
    \hat{\beta_k} = \left(
        \vec{y}^k -
        2\vec{d}^k
        \frac{\lVert\vec{y}^k\rVert^2}{\left(\vec{d}^k\right)^\mathsf{T}\cdot \vec{y}^k}
    \right)^\mathsf{T}
    \cdot 
    \frac{\vec{g}^{k+1}}{\left(\vec{d}^k\right)^\mathsf{T}\cdot\vec{y}^k}
\end{equation}
The vector $\vec{y}^k$ is defined as $\vec{y}^k=\vec{g}^{k+1}-\vec{g}^k$.
$\eta > 0$ is a user-defined constant to control the lower bound of allowed $\beta_k$.
$\hat{\beta_k}$ in \eqref{eq:hager-cg-beta-hat-k} may be negative, meaning we rewind somehow on the current search direction before starting a new search.
By limiting the lower bound, we can limit how much this rewinding effect can be.
Rewinding is not necessarily a bad phenomenon.
It can be seen as a restart on the search when the current search has no progress.
We use $\eta=0.01$ for all cases in this work.

As for the learning rate $\alpha_k$, theoretically, $\alpha_k$ is the value that makes $r(\vec{\theta}^k+\alpha_k\vec{d}^k)$ the absolute minimum in the direction $\vec{d}^k$.
A line search algorithm that finds such $\alpha_k$ is called an exact line search algorithm.
However, an exact line search is expensive.
Using exact line search hurts the performance advantage of CG (as well as gradient-descent methods).
It is, therefore, more common to use inexact line search algorithms.
In inexact line search, at the $k$-th iteration, we only seek an $\alpha_k$ that helps us land at a new location along $\vec{d}^k$ where
\begin{enumerate}[noitemsep,topsep=-12pt]
    \item we have a sufficient decrease in the loss and
    \item the slope in the $\vec{d}^k$ direction at the new location is milder than that at the beginning of the current iteration.
\end{enumerate}
Several mathematical conditions exist to evaluate these two textual statements quantitatively, including the Wolfe conditions, the strong Wolfe conditions, and the approximate Wolfe conditions.
We skip the mathematical details and expressions of the two conditions here to help focus on the meaning rather than the formulae.
Please refer to \cite[Chapter~3]{nocedal_numerical_2006} for mathematical details.

Alongside the CG variant \eqref{eq:hager-cg-beta}-\eqref{eq:hager-cg-beta-hat-k}, Hager and Zhang also proposed a companion inexact line search algorithm to find the learning rate $\alpha_k$ that helps this particular GC method to converge.
We have implemented both the CG and this line search in this work.
They will be used in some benchmarks to see if they can be beneficial to the training of data-free PINNs.
% vim:ft=tex


\section{Remarks on Differences and Similarities with Conventional Numerical Methods}\label{sec:pinn-diff}
%! TEX root = main.tex
In this section, we would like to make a brief analogy between traditional numerical methods and PINNs.
When solving differential equations numerically, we can describe the solution workflows of most numerical methods with five stages:
\begin{enumerate}[nolistsep]
    \item Designing the approximate solution with undetermined parameters
    \item Choosing proper approximation for derivatives
    \item Obtaining the modified equations by substituting approximate derivatives into the differential equations, IC, and BCs
    \item Generating a system of linear/nonlinear algebraic equations
    \item Solving the system of equations
\end{enumerate}

For example, to solve $\frac{\diff^2 U(x)}{\diff x^2}=s(x)$, the most naive spectral method \cite{trefethen_spectral_2000} approximates the solution with $U(x)\approx G(x; \Theta)\equiv\sum\limits_{i=1}^{N}\theta_i\phi_i(x)$, where $\theta_i$ represents the free model parameters; $\phi_i(x)$ denotes a set of either polynomials, trigonometric functions, or complex exponentials; and $N$ is the number of terms in the approximation.
Next, approximating the 1st-order derivative $\frac{\diff U(x)}{\diff x}$ is straightforward--we can assume $\frac{\diff U(x)}{\diff x} \approx \pdiff{G(x; \Theta)}{x}=\sum\limits_{i=1}^{N}\theta_i \frac{\diff \phi_i(x)}{\diff x}$.
The 2nd-order derivative may follow the same workflow: $\frac{\diff^2 U(x)}{\diff x^2} \approx \frac{\partial^2 G(x; \Theta)}{\partial x^2}=\sum\limits_{i=1}^{N}\theta_i \frac{\diff^2 \phi_i(x)}{\diff x^2}$.
$\phi_i(x)$ is known, so the derivatives $\frac{\diff \phi_i(x)}{\diff x}$ and $\frac{\diff^2 \phi_i(x)}{\diff x^2}$ are analytical.

Substitute the approximate derivatives into the differential equation, we obtain the residual function in continuous space: $r(x; \Theta) \equiv \sum\limits_{i=1}^{N}\theta_i \frac{\diff^2 \phi_i(x)}{\diff x^2} - s(x)$.

Finally, to determine the actual values of $\theta_i$, one approach is to use $N$ distinct $x$ values at which $r(x; \Theta) = 0$.
This results in a system of linear equations: 
\begin{equation}\label{eq:spectral-linear-sys}
    \begin{bmatrix}
        \frac{\diff^2 \phi_1}{\diff x^2}(x_1) & \cdots & \frac{\diff^2 \phi_N}{\diff x^2}(x_1) \\
        \vdots & \ddots & \vdots \\
        \frac{\diff^2 \phi_1}{\diff x^2}(x_N) & \cdots & \frac{\diff^2 \phi_N}{\diff x^2}(x_N) \\
    \end{bmatrix}
    \begin{bmatrix}
        \theta_1 \\ \vdots \\ \theta_N
    \end{bmatrix}
    - 
    \begin{bmatrix}
        s(x_1) \\ \vdots \\ s(x_N)
    \end{bmatrix}
    = 0
\end{equation}
Solving this linear system determines the values of $\theta_i$ and conclude the solving workflow of this naive spectral method.
The obtained $\theta_i$ guarantees that the residuals are zero at least on the $N$ chosen $x$ coordinates.

Though this example uses a spectral method, the workflow also applies to many other numerical methods, such as finite difference methods, which can be reformatted as a form of spectral method.

Alternatively, some numerical methods solve $\theta_i$ through finding the values that minimize the square of the residual across the whole $x$ domain:
\begin{equation}\label{eq:least-square-fem}
    \begin{aligned}
        \begin{bmatrix}
            \theta_1 \\ \vdots \\ \theta_N
        \end{bmatrix}
        & =
        \operatorname*{arg\,min}\limits_{\theta_i}
        \int\limits_{x}\left[r(x; \Theta)\right]^2\diff x \\
        & =
        \operatorname*{arg\,min}\limits_{\theta_i}
        \int\limits_{x}\left[\sum\limits_{i=1}^{N}\theta_i \frac{\diff^2 \phi_i(x)}{\diff x^2} - s(x)\right]^2\diff x
    \end{aligned}
\end{equation}
However, in these numerical methods, the optimization is done by solving the zero-slope conditions directly:
\begin{equation}
    \pdiff{\left[r(x; \Theta)\right]^2}{\theta_1} = 
    \pdiff{\left[r(x; \Theta)\right]^2}{\theta_2} = 
    \cdots =
    \pdiff{\left[r(x; \Theta)\right]^2}{\theta_N} = 
    0
\end{equation}
which also results in a linear system:
\begin{equation}\label{eq:spetral-least-equare}
    \begin{bmatrix}
        \int\limits_{x}
        \frac{\diff^2 \phi_1}{\diff x^2}
        \frac{\diff^2 \phi_1}{\diff x^2}
        \diff x
        &
        \cdots
        &
        \int\limits_{x}
        \frac{\diff^2 \phi_N}{\diff x^2}
        \frac{\diff^2 \phi_1}{\diff x^2}
        \diff x \\
        \vdots & \ddots & \vdots \\
        \int\limits_{x}
        \frac{\diff^2 \phi_1}{\diff x^2}
        \frac{\diff^2 \phi_N}{\diff x^2}
        \diff x
        &
        \cdots
        &
        \int\limits_{x}
        \frac{\diff^2 \phi_N}{\diff x^2}
        \frac{\diff^2 \phi_N}{\diff x^2}
        \diff x \\
    \end{bmatrix}
    \begin{bmatrix}
        \theta_1 \\ \vdots \\ \theta_N
    \end{bmatrix}
    - 
    \begin{bmatrix}
        \int\limits_{x}s(x)\frac{\diff \phi_1}{\diff x}(x) \diff x \\
        \vdots \\
        \int\limits_{x}s(x)\frac{\diff \phi_1}{\diff x}(x) \diff x
    \end{bmatrix}
    = 0
\end{equation}
As $\phi_i$ is given, the integrals can be evaluated analytically.
Moreover, a proper choice of $\phi_i$ can make the coefficient matrix in \eqref{eq:spetral-least-equare} sparse, making it cheap and fast to solve.
Finally, $\Theta=\{\theta_1,\cdots\theta_N\}$ is determined.
In this approach, the obtained $\theta_i$ guarantees that the integrated residual over the whole domain is minimal, but it does not guarantee a zero residual unless $r(x; \Theta) = \sum\limits_{i=1}^{N}\theta_i \frac{\diff^2 \phi_i(x)}{\diff x^2} - s(x)$ has zero roots with respect to $\Theta$ for any given $x$.

If we replace the integral in \eqref{eq:least-square-fem} with a Monte-Carlo numerical integration:
\begin{equation}\label{eq:least-square-monte-carlo}
    \begin{aligned}
        \begin{bmatrix}
            \theta_1 \\ \vdots \\ \theta_N
        \end{bmatrix}
        & =
        \operatorname*{arg\,min}\limits_{\theta_i}
        \int\limits_{x}\left[r(x; \Theta)\right]^2\diff x \\
        &\approx
        \operatorname*{arg\,min}\limits_{\theta_i}
        \frac{L_x}{N}\sum\limits_{m=1}^{N}\left[r(x_m; \Theta)\right]^2 \\
        & =
        \operatorname*{arg\,min}\limits_{\theta_i}
        \sum\limits_{m=1}^{N}\left[\sum\limits_{i=1}^{N}\theta_i \frac{\diff^2 \phi_i(x_m)}{\diff x^2} - s(x_m)\right]^2
    \end{aligned}
\end{equation}
where $L_x$ is the computational domain, and $N$ is the number of random points sampled from the computational domain.
We can see that \eqref{eq:least-square-monte-carlo} is actually the loss function in PINNs (equations \eqref{eq:residual-norms} and \eqref{eq:total-residual}) tailored to this specific example.
(The BCs are omitted from the beginning of this example for simplicity.)
Equations \eqref{eq:least-square-fem}-\eqref{eq:spetral-least-equare} are called least-square finite-element methods (least-square FEMs).
This shows PINNs can be seen as a type of global least-square FEMs.

With the spectral and least-square finite element methods in mind, we can see the analogy between PINNs and conventional numerical methods.
Especially when comparing to the least-square FEMs, their workflows are very close.
The main difference is that PINNs replace the analytical integration with a Monte-Carlo numerical integration.
Further, while the least-square FEMs solve zero-slope conditions to find the minimum residuals, PINNs rely on searching the minimal residuals on a complicated hypersurface of residuals iteratively.
It is due to the complexity and the nonlinearity of the models in PINNs, where neither solving zero-residual conditions on selected points nor solving zero-slope conditions can reduce problems to a linear system. 

The second difference is how to approximate derivatives.
Conventional numerical methods use analytical or numerical differentiation of the models to approximate derivatives.
On the other hand, modern PINNs depend on automatic differentiation due to the complexity and nonlinearity in the networks.

The two differences play an important role in solution accuracy and time-to-solution.
Solving linear systems is intuitively faster than solving a nonlinear optimization problems through 1st-order optimization methods and is cheaper than 2nd-order methods in terms of memory usage.
And while automatic differentiation is powerful, it requires a nontrivial computer memory.
As seen in section \ref{sec:ad}, automatic differentiation needs to memorize all computational graphs for derivatives.
And high-order derivatives needs also to memorize the computational graphs of lower-order derivatives, making the computational graph to grow exponentially.

% vim:ft=tex


\section{Code Implementation}\label{sec:pinn-code-impl}
%! TEX root = main.tex
The PINN solvers used in the later benchmarks were implemented with the help of NVIDIA's Modulus \cite{noauthor_modulus_nodate}.
Modulus utilizes symbolic expression, constructive solid geometry, computational graphs, and PyTorch to make defining a PINN PDE solver easier.
It provides some pre-defined PDE solvers and neural network architectures.
Regular users only need to configure the computational domains and constraint instances to evaluate each loss term in equation \eqref{eq:total-residual-weighted}.
(For example, the number of training points per batch, sampling regions, or weights.)
These configurations are done with a combination of YAML files and Python code.
For simple use cases, the Python code only consists of several lines defining how to interpret the key-value pairs in the YAML files and calling APIs to set up loss terms.

Another feature of Modulus is using SymPy's symbolic expression \cite{meurer_sympy_2017} as a user interface.
For example, if users want to sample points from the region where $x > 0.5$, they add a SymPy symbolic expression instance of \lstinline{sympy.Gt(x, 0.5)} to the corresponding constraint instance.
Internally, Modulus does not rely on symbolic calculation.
Instead, it translates users' symbolic expressions to numeric calculations in PyTorch.
With this symbolic user interface, users are even allowed to define new PDEs by using symbolic expressions.
Please refer to Modulus' manual for details.

In our work, we further customized and added several components in Modulus to meet our needs:
\begin{enumerate}[nolistsep]
    \item A reimplementation of the unsteady incompressible Navier-Stokes equations.
    Modulus also ships a similar class for the compressible Navier-Stokes equations, which falls back to the same incompressible formulation when the density is a constant.
    We reimplemented it for a better control over the source code.
    \item Other non-governing PDEs, such as vorticity, Q criterion, and Neumann and convective boundary conditions.
    \item A binding to PyTorch's \lstinline{CylicLR} learning schedule.
    \item Nonlinear CG and inexact line search as a subclass of \lstinline{torch.optim.Optimizer}.
    \item Reimplementation of the annealing loss aggregation based on equation \eqref{eq:annealing-in-this-work}. The original annealing loss aggregation implementation was hard-coded for single-variable PDEs, which does not work with PDE systems like the Navier-Stokes equations.
    \item A solution workflow controller that allows the use of annealing loss aggregation, Adam, nonlinear CG, and SWA in an arbitrary combination. 
    \item An inferencer that allows to output only the desired portion of a PINN model every given iterations. 
    \item Boundary and interior constraint classes that are aware of the counter of training iterations so that they keep re-using the same batch of training points when the iteration counter remains the same.
    They are used when the CG optimizer and the annealing loss aggregation are both enabled.
\end{enumerate}

The code for the mentioned new/reimplemented components is available at \cite{chuang_dissertation_nodate}.
The same reference also contains all other code, scripts, and the container definition files for reproducing the results and figures in the later benchmarks.
% vim:ft=tex
% vim:ft=tex:
