%! TEX root = main.tex
\chapter{PINNs and Details of the Methods Used in This Work}\label{chap:pinn}

\section{Solution Workflow}\label{sec:pinn-workflow-overview}

    \subsection{Overview}\label{sec:pinn-overview}
    %! TEX root = main.tex

The basic form of the PINN method (\cite{raissi_physics-informed_2019,cai_physics-informed_2021}) starts from approximating $\vec{U}$ and $p$ with a neural network:
\begin{equation}\label{eq:neural-network}
    \begin{bmatrix}
        \vec{U} \\ p
    \end{bmatrix}(\vec{x}, t)
    \approx
    G(\vec{x}, t; \Theta)
\end{equation}
Here we use a single network that predicts both pressure and velocity fields.
It is also possible to use different networks for them separately.
Later in this work, we will use $G^U$ and $G^p$ to denote the predicted velocity and pressure from the neural network.
$\Theta$ at this point represents the free parameters of the network.

To determine the free parameters, $\Theta$, ideally, we hope the approximate solution gives zero residuals for equations (\ref{eq:continuity}), (\ref{eq:momentum}), and (\ref{eq:ic-and-bc}).
That is

\begin{equation}\label{eq:residuals}
    \begin{aligned}
        & r_{1}(\vec{x}, t; \Theta) \equiv \nabla \cdot G^{U} = 0 \\
        & r_{2}(\vec{x}, t; \Theta) \equiv \frac{\partial G^{U}}{\partial t}+(G^{U} \cdot \nabla) G^{U}+\frac{1}{\rho} \nabla G^p -\nu \nabla^{2} G^{U} - \vec{g} =0 \\
        & r_{3}(\vec{x}; \Theta) \equiv G^{U}_{t=0}-\vec{U}_0 = 0 \\
        & r_{4}(\vec{x}, t; \Theta) \equiv G^{U}-\vec{U}_\Gamma = 0,\ \forall \vec{x} \in \Gamma \\
        & r_{5}(\vec{x}, t; \Theta) \equiv G^{p}-p_\Gamma = 0,\ \forall \vec{x} \in \Gamma \\
   \end{aligned}
\end{equation}

And the set of desired parameter, $\Theta=\theta$, is the common zero root of all the residuals.

The derivatives of $G$ with respect to $\vec{x}$ and $t$ are usually obtained using automatic differentiation. 
Nevertheless, it is possible to use analytical derivatives when the chosen network architecture is simple enough, as reported by early-day literature \cite{lagaris_artificial_1998,Li2003}.

If residuals in (\ref{eq:residuals}) are not complicated, and if the number of the parameters, $N_\Theta$, is small enough, we may numerically find the zero root by solving a system of $N_\Theta$ nonlinear equations generated from a suitable set of $N_\Theta$ spatial-temporal points.
However, the scenario rarely happens as $G$ is usually highly complicated and $N_\Theta$ is large.
Moreover, we do not even know if such a zero root exists for the equations in (\ref{eq:residuals}).

Instead, in PINN, the condition is relaxed.
We do not seek the zero root of (\ref{eq:residuals}) but just hope to find a set of parameters that make the residuals sufficiently close to zero.
Consider the sum of the $l_2$ norms of residuals:

\begin{equation}\label{eq:total-residual}
    r(\vec{x}, t; \Theta=\theta) \equiv \sum\limits_{i=1}^{5} \lVert r_i(\vec{x}, t; \Theta=\theta) \rVert^2,\ \forall \left\{\begin{array}{l}x \in \Omega \\ t\in[0, T]\end{array}\right.
\end{equation}

The $\theta$ that makes residuals closest to zero (or even equal to zero if such $\theta$ exists) also makes (\ref{eq:total-residual}) minimal because $r(\vec{x}, t; \Theta) \ge 0$.
In other words,

\begin{equation}\label{eq:objective}
    \theta = \operatorname*{arg\,min}\limits_{\Theta} r(\vec{x}, t; \Theta)\,\ \forall \left\{\begin{array}{l}x \in \Omega \\ t\in[0, T]\end{array}\right.
\end{equation}

This poses a fundamental difference between the PINN method and traditional CFD schemes, making it potentially more difficult for the PINN method to achieve the same accuracy as the traditional schemes.
We will discuss this more in section 3.
Note that in practice, each loss term on the right-hand-side of equation (\ref{eq:total-residual}) is weighted.
We ignore the weights here for demonstrating purpose. 

To solve (\ref{eq:objective}), theoretically, we can use any number of spatial-temporal points, which eases the need of computational resources, compared to finding the zero root directly.
Gradient-descent-based optimizers further reduce the computational cost, especially in terms of memory usage and the difficulty of parallelization.
Alternatively, Quasi-Newton methods may work but only when $N_\Theta$ is small enough.

However, even though equation (\ref{eq:objective}) may be solvable, it is still a significantly expensive task.
While typical data-driven learning requires one back-propagation pass on the derivatives of the loss function, here automatic differentiation is needed to evaluate the derivatives of $G$ with respect to $\vec{x}$ and $t$.
The first-order derivatives require one back-propagation on the network, while the second-order derivatives present in the diffusion term $\nabla^2 G^U$ require an additional back-propagation on the first-order derivatives' computational graph. 
Finally, to update parameters in an optimizer, the gradients of $G$ with respect to parameters $\Theta$ requires another back-propagation on the graph of the second-order derivatives.
This all leads to a very large computational graph.
We will see the performance of the PINN method in the case studies.

In summary, when viewing the PINN method as supervised machine learning, the inputs of a network are spatial-temporal coordinates, and the outputs are the physical quantities of our interest.
The loss or objective functions in PINN are governing equations that regulate how the target physical quantities should behave. 
The use of governing equations eliminates the need for true answers.
A trivial example is using Bernoulli's equation as the loss function, i.e., $loss=\frac{u^2}{2g}+\frac{p}{\rho g}-H_0+z(x)$, and a neural network predicts the flow speed $u$ and pressure $p$ at a given location $x$ along a streamline.
(The gravitational acceleration $g$, density $\rho$, energy head $H_0$, and elevation $z(x)$ are usually known and given.)
Such a loss function regulates the relationship between predicted $u$ and $p$ and does not need true answers for the two quantities.
Unlike Bernoulli’s equation, most governing equations in physics are usually differential equations (e.g., heat equations).
The main difference is that now the PINN method needs automatic differentiation to evaluate the loss.
Regardless of the forms of governing equations, spatial-temporal coordinates are the only data required during training.
Hence, throughout this paper, training data means spatial-temporal points and does not involve any true answers to predicted quantities.
(Note in some literature, the PINN method is applied to applications that do need true answers, see \cite{cai_physics-informed_2021}. These applications are out of scope here.)

\begin{equation}\label{eq:silu}
    \operatorname{SiLU}(x)=\frac{x}{1+\exp({-x})}
\end{equation}

% vim:ft=tex


    \subsection{Batched Training}\label{sec:batched-training}
    %! TEX root = main.tex

Compared to solving $N_\Theta$ nonlinear equations directly, an optimization problem of \eqref{eq:total-residual} or \eqref{eq:total-residual-weighted} allows us to use any numbers of spatial-temporal points.
That is, there is no limitation on the magnitude of $N_{PDE}$, $N_{BC}$, and $N_{IC}$.
This eases the need of computational resources.
In data-free PINNs, these spatial-temporal points are called training points or training data.
They are spatial-temporal coordinates where we evaluate the residuals of PDEs, IC, and BCs.
While training points may be generated manually and intentionally positioned at desired locations (just like meshing in conventional CFD simulations), it is more common to generate them by selecting coordinates randomly using a uniform density function. 

The ability to use arbitrary numbers of training points helps in the generalizability of a trained PINNs model.
When more training points are involved in \eqref{eq:residual-norms} and hence \eqref{eq:total-residual-weighted}, the minimal residuals are achieved at more locations in the domain. 
Ultimately, with unlimited training points evenly distributed in the domain, the optimization of the discretized residuals in \eqref{eq:residual-norms} becomes optimization of the continuous residual functions in \eqref{eq:residuals}. 
And the resulting optimal parameters, $\vec{\theta}$, will make $G$ a more accurate approximation solution to the whole domain of $\Omega$ and $t\in[0, T]$.

More points also mean more computational cost.
To reduce the cost, a batched training is utilized.
For example, if using an iterative optimization method (such as the gradient-descent or Adam optimizer), we can always generate a new batch of training points for a new iteration to evaluate the discretized residuals.
After a significant number of iterations, the total number of training points involved in optimization will also be significant.
Once the model is trained, theoretically, it is therefore capable of giving accurate predictions at any locations and times.

In practice, however, it is not a cheap task if thousands or even more random points have to be generated on the fly at each iteration.
We instead generated a fixed amount of training points before the optimization and only used a batch of them in each iteration.
For example, to use $N_{PDE}$ points to evaluate the PDE residuals at each iteration, we may generate $N_{PDE}\times 1000$ points in advance and divide them into $1000$ batches.
If we run an optimizer for 1 million iterations, each batch is repeated every 1000 iterations.

Theoretically, each batch should have similar statistical properties and hence give a similar gradient vector under fixed model parameters.
That is
\begin{equation}
\left.\nabla_\theta r(\Theta) \right|_{\vec{x}, t \in \mathcal{X}_1} \sim
\left.\nabla_\theta r(\Theta) \right|_{\vec{x}, t \in \mathcal{X}_2} \sim
\cdots
\end{equation}
where $\mathcal{X}_i$ for $i=1,2,\cdots$ represents the $i$-th batch of the training points.
These gradients are used to update $\vec{\theta}$ in iterative optimization methods.
In other words, the hypersurface of $r(\Theta)$ is expected not to change significantly from iteration to iteration when using different batches of points to evaluate it.
However, whether this statement is true may depend on the number of points in each batch: the number of points should be big enough to have similar statistical properties across different batches.

In PINNs, as the training points are randomly sampled from the computational space-time domain, it means the number of points in a batch should be big enough to cover all over the domain.
Otherwise, for example, if one batch mostly covers the inflow region of a cylinder flow, while the next batch covers mostly the wake region behind the cylinder, then the hypersurface of the PDE residuals will change significantly.
And this change of the hypersurface from iteration to iteration may slow down the convergence.
In the later benchmarks, we would like to examine the effect of batch sizes, i.e., number of points in each batch. 

% vim:ft=tex

    \subsection{Adaptive Loss Aggregation}\label{sec:loss-annealing}
    %! TEX root = main.tex
In equation \eqref{eq:total-residual-weighted}, each loss term is weighted.
How to properly assign the weights is still an open question.
In references \cite{jin_nsfnets_2020,wang_understanding_2021}, the authors proposed an annealing approach to change these weights adaptively during the optimization process.
However, these adaptive approaches have not been further tested by more works.
In this work, part of the benchmarks will cover the performance of the adaptive weight strategy proposed in \cite{jin_nsfnets_2020}.

Jin et al. \cite{jin_nsfnets_2020} proposed the following annealing loss aggregation algorithm for iterative optimization methods:
\begin{equation}
    r^k(\Theta) = r_{PDE}^k(\Theta) + 
        \left(\left(1-\lambda\right)\alpha^k + \lambda\alpha^{k+1}\right)r_{BC}^k(\Theta) + 
        \left(\left(1-\lambda\right)\beta^k + \lambda\beta^{k+1}\right)r_{IC}^k(\Theta)
\end{equation}
where $k$ denotes the $k$-th iteration in an iterative optimization method.
The subscript $PDE$, $BC$, and $IC$ denote the loss contributed by the residuals of PDEs, BCs, and IC.
$\lambda$ is a user-provided parameter to control the moving average of the current and the previous weights.
The concept of this adaptive approach is to make the gradients of each loss term comparable.
The magnitude of each loss term is reduced at a similar rate when using the gradient-descent method and its derived methods.
And
\begin{equation}
    \alpha^{k+1} = \frac{\overline{\lvert\nabla_\Theta r_{PDE}^k(\Theta)\rvert}}{\overline{\lvert\nabla_\Theta r_{BC}^k(\Theta)\rvert}}
    \text{\ \ \ \ and\ \ \ \ }
    \beta^{k+1} = \frac{\overline{\lvert\nabla_\Theta r_{PDE}^k(\Theta)\rvert}}{\overline{\lvert\nabla_\Theta r_{IC}^k(\Theta)\rvert}}
\end{equation}
$\lvert\cdot\rvert$ denotes the element-wise absolute values of a vector.
$\overline{\lvert\cdot\rvert}$ is the mean of these absolute values.

The following expression better represents the actual implementation in our code:
\begin{equation}
    \begin{aligned}
        &\zeta^k =
            \overline{\lvert\nabla_\Theta r_{m,u}^k(\Theta)\rvert} +
            \overline{\lvert\nabla_\Theta r_{m,v}^k(\Theta)\rvert} +
            \overline{\lvert\nabla_\Theta r_{m,w}^k(\Theta)\rvert} +
            \overline{\lvert\nabla_\Theta r_{c}^k(\Theta)\rvert} \\
        &\alpha^{k+1} = 
            \frac{\zeta^k}{\overline{\lvert\nabla_\Theta r_i^k(\Theta)\rvert}} \\
        &r^k(\Theta) = r_m^k(\Theta) + r_c^k(\Theta) + 
            \sum\limits_{i} \left(\left(1-\lambda\right)\alpha_i^k + \lambda\alpha_i^{k+1}\right) r_i^k(\Theta)
    \end{aligned}
\end{equation}
where $r_{m,u}^k$, $r_{m,v}^k$, and $r_{m,w}^k$ are the $u$-, $v$-, and $w$-component in the residual vector of the momentum equations at $k$-th iteration.
And subscript $i$ denotes different loss terms, excluding $r_m(\Theta)^k$ and $r_c(\Theta)^k$.
In this work, $\lambda$ is fixed at $0.1$ for all cases using annealing loss aggregation.
% vim:ft=tex

\section{Deep Neural Network Modeling}\label{sec:pinn-dnnm}

    \subsection{Multilayer Perceptron Networks}\label{sec:mlp}
    %! TEX root = main.tex

In the previous section, $G$ denotes a neural network model (or any mathematical model, actually) that predicts flow quantities at given spatial-temporal coordinates.
The exact mathematical expression of $G$ was omitted.
In this section, we would like to introduce the mathematical expression of the most common neural network model in data-free PINNs: MLP (multilayer perceptron) networks.

The universal approximation theorem \cite{hornik_approximation_1991} states that large-enough MLP networks can approximate any smooth functions.
This theorem justifies the use of MLP networks as PDEs' approximation solutions.
These PDEs include the Navier-Stokes equations, which are the major governing equations for CFD.\footnote[0]{Though the existence of the smooth solutions for the Navier-Stokes equations are not yet proved, as CFD practitioners, we usually ignore this theoretical fact.}

Figure \ref{fig:mlp-graph} is a commonly seen graphical illustration for how MLP works in literature.
To avoid introducing more mathematical symbols, in this section and in figure \ref{fig:mlp-graph}, we reuse the notation from the previous section.

\begin{figure}
    \singlespacing
    \includegraphics[width=0.95\linewidth]{figs/mlp.tikz}
    \caption{Graphical illustration of MLP networks}
    \label{fig:mlp-graph}
\end{figure}

An MLP network fundamentally is a series of linear-nonlinear mapping pairs:
\begin{equation}\label{eq:mlp-formula}
    \begin{array}{ll}
        \vec{h}^0 \equiv \begin{bmatrix} \vec{x} \\ t \end{bmatrix} & \\
        \vec{h}^k = \sigma_{k-1}\left(A^{k-1}\vec{h}^{k-1}+\vec{b}^{k-1}\right)\text{,} & 1 \le k \le N_l \\
        \vec{h}^{N_l+1}\equiv \begin{bmatrix} G^{\vec{u}} \\ G^p \end{bmatrix} = \sigma_{N_l}\left(A^{N_l}\vec{h}^{N_l}+\vec{b}^{N_l}\right) &
    \end{array}
\end{equation}
$\vec{h}^k$ is called a hidden layer or a latent space vector.
$\sigma_{k}$ is a chosen element-wise nonlinear function.
$A^{k}$ and $\vec{b}^k$ for $k=0$ to $k=N_l$ are parameter matrices and vectors that consist of free model parameters.
In other words, $\Theta=\left\{A^0, \vec{b}^0, A^1, \vec{b}^1, \cdots \right\}$.
$N_l$ denotes the number of hidden layers, and $N_n$ in figure \ref{fig:mlp-graph} represents number of elements in each hidden layer, $\vec{h}^k$.
Elements in $\vec{h}^k$ are often called neurons in deep learning.
Theoretically, $N_n$ does not have to be constant.
Each $\vec{h}^k$ can have different $N_n$.
However, as there is no clear guideline on specifying variate $N_n$ for PINNs, most reports in PINNs just used a constant $N_n$ across hidden layers.
We follow the same approach in this work.

$A^k$ and $\vec{b}^k$ are the free parameters we want to determine in the optimization step (i.e., equation \eqref{eq:total-residual-weighted}).
$N_l$ and $N_n$ are also parameters that control the complexity and accuracy of an MLP network.
However, they are determined by users rather than by optimization.
This type of parameters is called hyperparameters--they are not counted into the degree of freedom of a mathematical model.

The element-wise nonlinear functions $\sigma_{k}$ are also pre-determined by users.
Theoretically, they can be different functions across different $k$.
To our knowledge, most reports in PINNs used the same function for all $k$, i.e., $\sigma_0(x)=\sigma_1(x)=\cdots=\sigma(x)$.
In PINNs, most commonly seen choices of $\sigma$ are hyperbolic tangent ($\tanh$) and the sigmoid function:
\begin{equation}
    \operatorname{sigmoid}(x) = \frac{1}{1+\exp(-x)}
\end{equation}
In this work, we instead use SiLU for $\sigma$ \cite{hendrycks_gaussian_2020}, which is the default option for Modulus:
\begin{equation}\label{eq:silu}
    \operatorname{SiLU}(x) = \frac{x}{1+\exp(-x)}
\end{equation}
To our best knowledge, in terms of solving flow problems with data-free PINNs, no reports have discussed the effect of $\sigma$ except for \cite{li_integration_2010}.

While simple MLP networks like \eqref{eq:mlp-formula} are not popular in modern deep learning applications, they are still the first choice in solving PDEs with data-free PINNs.
The reason may be that MLP networks are supported by the universal approximation theorem, and most other networks were not.
The universal approximation theorem proves MLP networks' ability to approximate any smooth functions, which makes them theory-backed mathematical models for PDEs' general solutions.
Nevertheless, lacking theoretical proofs does not mean other networks would not work.
For example, \cite{sirignano_dgm:_2018} reported the success of using an LSTM-style network to solve PDEs.

Networks with higher complexity imply more expensive computational cost.
Whether using complicated networks is worth it may be another open question in terms of cost-performance ratios. 
Instead, in this work, we used a variant of MLP networks implemented in Modulus--a weight-normalization MLP \cite{salimans_weight_2016}:
\begin{equation}\label{eq:weighted-norm-mlp-formula}
    \begin{array}{ll}
        \vec{h}^0 \equiv \begin{bmatrix} \vec{x} \\ t \end{bmatrix} & \\
        \vec{h}^k = \sigma_{k-1}\left(
            \operatorname{diag}\left(\vec{w}_g^{k-1}\right)
            \operatorname{diag}\left(\vec{w}_n^{k-1}\right)^{-1}
            W^{k-1}\vec{h}^{k-1}+\vec{b}^{k-1}
        \right)\text{,} & 1 \le k \le N_l \\
        \vec{h}^{N_l+1}\equiv \begin{bmatrix} G^{\vec{u}} \\ G^p \end{bmatrix} = \sigma_{N_l}\left(A^{N_l}\vec{h}^{N_l}+\vec{b}^{N_l}\right) &
    \end{array}
\end{equation}
$\vec{w}_g^{k}$ for $k=0,\cdots,N_l$ represents a parameter vector.
$W^k$ is a parameter matrix.
$\vec{w}_n^{k}$ is not a parameter vector but holds the norms of each row in $W^k$.
The key idea is to decouple each row in $A^k$ to a length multiplying a direction.
$\operatorname{diag}\left(\vec{w}_n^k\right)W^k$ is a row-normalized matrix, that is, each row is a unit vector.
$\vec{w}_g^k$ represents the lengths of the corresponding rows.

Equation \eqref{eq:weighted-norm-mlp-formula} does not actually alter the architecture of an MLP network but rearranges the mathematical expression.
Theoretically, either using \eqref{eq:mlp-formula} or \eqref{eq:weighted-norm-mlp-formula}, a perfect optimization scheme should find $A^k$ and $\operatorname{diag}\left(\vec{w}_g^{k-1}\right) \operatorname{diag}\left(\vec{w}_n^{k-1}\right)^{-1} W^{k-1}$ equivalent.
However, real-world is far from perfect, and Salimans and Kingma \cite{salimans_weight_2016} found using the expression of \eqref{eq:weighted-norm-mlp-formula} speeds up the convergence when using 1st-order optimization schemes, such as the basic gradient-descent and Adam.

The weight-normalization variant of the MLP networks has the following formula to calculate the total number of free parameters:
\begin{equation}\label{eq:dof-calculator}
    DoF =
     N_{n} \left(N_{in} + 2\right) + 
    N_{n} \left(N_{n} + 2 \right) \left(N_l-1\right) +
     N_{out}\left(N_{n} + 1\right)
\end{equation}
where $DoF$ means the degree of freedom, which is another term for the number of free parameters.
% vim:ft=tex


    \subsection{Handling of Periodic Boundaries}\label{sec:periodic-boundary}
    %! TEX root = main.tex

When dealing with periodic boundary conditions, to calculate the loss (i.e., $\vec{r}_{bc,\vec{u}}$ and $\vec{r}_{bc, p}$), one possible approach is to define the loss terms as the difference in the values between the pair of the corresponding boundaries.
For example, assume we have a periodic BC on $x=0$ and $x=L$.
We define
\begin{equation}\label{eq:naive-periodic-bc}
    \begin{aligned}
    &r_{bc,\vec{u}}(\Theta) = \sum\limits_{i=1}^{N_{BC}} \lVert G^{\vec{u}}(\vec{x}_{0,i}, t_i; \Theta) - G^{\vec{u}}(\vec{x}_{L,i}, t_i; \Theta) \rVert^2\\
    &r_{bc,p}(\Theta) = \sum\limits_{i=1}^{N_{BC}} ( G^{p}(\vec{x}_{0,i}, t_i; \Theta) - G^{p}(\vec{x}_{L,i}, t_i; \Theta) )^2
    \end{aligned}
\end{equation}
where $\vec{x}_{0,i}$ and $\vec{x}_{L,i}$ are spatial coordinates on the plane of $x=0$ and $x=L$, respectively.
And the coordinate components of $y$ and $z$ in $\vec{x}_{0,i}$ and $\vec{x}_{L,i}$ should match.

Modulus provides an alternative approach.
Consider a pair of periodic BCs on $x=x_1$ and $x=x_2$.
We notice that $\sin(2\pi\frac{x-x_1}{x_2-x_1})$ and  $\cos(2\pi\frac{x-x_1}{x_2-x_1})$ have a period of $x_2-x_1$.
Hence, we expand the inputs of $G$ to
\begin{equation}\label{eq:periodic-G}
    G = G(\sin(2\pi\frac{x-x_1}{x_2-x_1}), \cos(2\pi\frac{x-x_1}{x_2-x_1}), y, z, t; \Theta)
\end{equation}
If component $y$ or $z$ also has periodic BCs, they are converted as well, following the same logic.
In the present work, the periodic BCs are handled using this approach.
This approach builds the periodicity into the model directly; hence no BC loss terms are needed for periodic BCs.
% vim:ft=tex


    \subsection{Automatic Differentiation}\label{sec:ad}
    %! TEX root = main.tex
The PINN method relies on automatic differentiation to evaluate the derivatives of $G$.
Automatic differentiation algorithms record how a numeric value is calculated and then apply the chain rule to determine that value's derivatives with respect to its inputs.
For example, assume a piece of code evaluates the following calculation:
\begin{equation}\label{eq:graph-example}
 y = \left.{e^{\left(x_1x_2-\sin{\left(x_2\right)}\right)}+3x_1}\right|_{\begin{subarray}{l}x_1=1.2 \\ x_2=0.5\end{subarray}} = 4.72\dots
\end{equation}
On computers, the calculation is realized by a sequence of basic unary and binary operations on numeric values.
Automatic differentiation algorithms record such a sequence in a computational graph, as illustrated in figure~\ref{fig:automatic-differentiation-forward}.
\begin{figure}[hbt!]
    \Centering
    \begin{minipage}[c]{0.4\textwidth}
        \includegraphics[width=\linewidth]{figs/automatic-differentiation-forward.tikz}
    \end{minipage}%
    \begin{minipage}[c]{0.5\textwidth}
        \scriptsize
        \singlespacing
        \begin{equation*}
            \begin{aligned}
                &v_0 = \mathrm{assign}\left(1.2\right) \\
                &v_1 = \mathrm{assign}\left(0.5\right) \\
                &v_2 = \mathrm{assign}\left(3\right) \\
                &v_3 = \mathrm{multiply}\left(v_2, v_0\right)  = 3 \times 1.2 = 3.6 \\
                &v_4 = \mathrm{multiply}\left(v_0, v_1\right) = 1.2 \times 0.5 = 0.6 \\
                &v_5 = \sin\left({v_1}\right) = \sin\left(0.5\right) = 0.479\dots \\
                &v_6 = \mathrm{negate}\left(v_5\right) = -0.479\dots \\
                &v_7 = \mathrm{add}\left(v_4, v_6\right) = 0.12\dots \\
                &v_8 = \exp\left({v_7}\right) = 1.128\dots \\
                &v_9 = \mathrm{add}\left(v_3, v_8\right) = 4.728\dots
            \end{aligned}
        \end{equation*}
    \end{minipage}
    \caption{Computational graph of equation~\ref{eq:graph-example}}%
    \label{fig:automatic-differentiation-forward}
\end{figure}

It is relatively easy to evaluate the derivatives of unary and binary functions with respect to their inputs.
Combining this fact and the chain rule, automatic differentiation evaluates:
\begin{equation}\label{eq:reverse-graph-example}
    \left\{
        \begin{array}{ll}
            d_9 = \pdiff{v_9}{v_9} = 1 & \\
            d_i = \pdiff{v_9}{v_i}
                = \sum\limits_{p\in P(i)}\pdiff{v_9}{v_p}\pdiff{v_p}{v_i}
                = \sum\limits_{p\in P(i)}d_p\pdiff{v_p}{v_i}, & \text{for }i=8, 7, \ldots, 1, 0
        \end{array}
    \right.
\end{equation}
The set $P(i)$ denotes the parents of node $i$ in the reversed computational graph.
For example, as shown in figure~\ref{fig:automatic-differentiation-backward}, the parents of node $1$ are $4$ and $5$.
Any derivatives with respect to $v_1$ have contributions from nodes $4$ and $5$.
Note that we evaluate equation~\ref{eq:reverse-graph-example} in a reversed order, $i=9, 8, 7, \ldots$, so that $d_p$ ($\forall p\in P(i)$) are ready when evaluating any $d_i$.
See figure~\ref{fig:automatic-differentiation-backward} for the backward calculation.
\begin{figure}[hbt!]
    \begin{minipage}{0.4\textwidth}
        \includegraphics[width=\linewidth]{figs/automatic-differentiation-backward.tikz}
    \end{minipage}%
    \begin{minipage}{0.5\textwidth}
        \scriptsize
        \singlespacing
        \begin{equation*}
            \begin{aligned}
                d_9& = \pdiff{v_9}{v_9} = 1 & d_8& = d_9 \pdiff{v_9}{v_8} = 1 \\[-2pt]
                d_7& = d_8 \pdiff{v_8}{v_7} = 1.128\dots & d_6& = d_7 \pdiff{v_7}{v_6} = 1.128\dots \\[-2pt]
                d_5& = d_6 \pdiff{v_6}{v_5} = -1.128\dots & d_4& = d_7 \pdiff{v_7}{v_4} = 1.128\dots \\[-2pt]
                d_3& = d_9 \pdiff{v_9}{v_3} = 1 & d_2& = d_3 \pdiff{v_3}{v_2} = 1.2
            \end{aligned}
        \end{equation*}%
        \begin{equation*}
            \begin{aligned}
                d_1& = d_4 \pdiff{v_4}{v_1} + d_5 \pdiff{v_5}{v_1} = d_4v_0 + d_5\cos(v_1) = 0.36\dots = \pdiff{y}{x_2} \\
                d_0& = d_4 \pdiff{v_4}{v_0} + d_3 \pdiff{v_3}{v_0} = d_4v_1+d_3v_2 = 3.56\dots = \pdiff{y}{x_1}
            \end{aligned}
        \end{equation*}
    \end{minipage}
    \caption{Computational graph of automatic differentiation of equation~\ref{eq:graph-example}}%
    \label{fig:automatic-differentiation-backward}
\end{figure}

Given that $v_9=y$, $v_0=x_1$ and $v_1=x_2$, once the backward calculation reaches $i=0$ and $i=1$, we automatically obtain $\pdiff{y}{x_1}$ and $\pdiff{y}{x_2}$.
Please refer to~\cite{griewank_automatic_1988} for a detailed introduction to automatic differentiation algorithms.

No matter how complicated a mathematical expression is (like a neural network $G$), it is always broken down to a sequence of simple binary or unary operations.
And hence automatic differentiation guarantees the exact derivatives in terms of this sequence of unary and binary computation.

The automatic differentiation is used to evaluate the derivatives of $G$'s outputs with respect to its inputs.
Also, it is used to evaluate the derivatives of the residuals with respect to model parameters, i.e., $\nabla_{\Theta} r(\Theta)$.
The latter is used by optimizers in the optimization process.
% vim:ft=tex

\section{Other Training Strategies}
%! TEX root = main.tex
The later benchmarks of PINNs in this work include some training strategies that have not been reported in literature of PINNs.
These strategies represent some latest development in other deep learning applications beyond PINNs.
We would like to see if these strategies help improve the accuracy or reduce computational cost of data-free PINNs.
% vim:ft=tex

    \subsection{Cyclical Learning Rates and Stochastic Weight Averaging}\label{sec:cyclic-swa}
    %! TEX root = main.tex
In iterative optimization methods that are derived from the gradient -descent method, the last step in each iteration is to update parameters with the corrected or uncorrected gradients of the loss with respect to model parameters:
\begin{equation}
    \vec{\theta}^{k+1} = \vec{\theta}^{k} - \eta\Phi(\nabla_{\Theta}r(\Theta=\vec{\theta}^k))
\end{equation}
where $\eta$ is a hyperparameter controlling the step size to move in the negative gradient direction on the hypersurface of $r(\Theta)$ in $\Theta$ space.
$\Phi(\nabla_{\Theta}r(\vec{\theta}^k))$ represents either a corrected or uncorrected gradient.
For example, $\Phi(\nabla_{\Theta}r(\vec{\theta}^k))=\nabla_{\Theta}r(\vec{\theta}^k)$ in the vanilla gradient-descent method.
And $\Phi(\nabla_{\Theta}r(\vec{\theta}^k)) = \left(1-\beta\right)\nabla_{\Theta}r(\vec{\theta}^k) + \beta \vec{g}^{k-1}$ for the gradient-descent with momentum, where $\vec{g}^{k-1}$ is the running average of $\nabla_{\Theta}r(\Theta)$ up to $k-1$-th iteration.
See \cite[Section~8.3]{goodfellow_deep_2016} for the details of these optimization methods.
As the iteration approaches a local minimum, the gradient approaches zero, and eventually the training converges to $\vec{\theta}$ when $\vec{\theta}^k=\vec{\theta}^{k-1}$

The step size $\eta$ is called {\it learning rate} in modern machine learning.
The configuration of the learning rate has always been an open question.
In earlier iterations, a large learning rate helps speed up the convergence.
However, in later iterations, when the training approach a local minimum, a large learning rate may make the step too big and miss the minimum.
So common practice is to have smaller and smaller learning rates for later iterations.
A way to adjust learning rate without human intervention is to use a learning rate scheduling based on the iteration counter.
For example, an exponential learning has a formula of
\begin{equation}
    \eta(k) = \eta_0 \times \gamma^k
\end{equation}
where $k$ denotes the $k$-th iteration, and $\gamma$ is a hyperparameter controlling the decaying rate.

Though a large learning rate may miss a local minimum, it sometimes also helps in escaping from a local minimum or a saddle point on the hypersurface of $r(\Theta)$.
With exponential learning rate adjustment, later iterations may lose the ability to find a new and better local minimum or lose the ability to quickly go over the plateaus surrounding saddle points.

Smith \cite{smith_cyclical_2017} proposed a cyclical learning rate adjusting strategy for general deep learning applications.
He found the cyclical learning rate increased the accuracy of a trained model and accelerated the convergence of the training.
In the cyclical learning rate, learning rate bounces back and forth between an upper and lower bounds.
The upper bound may be a constant or a value depending on the iteration counter.

The intuitive reason why cyclical learning rate works is that the difficulty of training deep neural networks is dominated by saddle points rather than poor local minimums \cite{dauphin_identifying_2014}.
The cyclical learning rate helps escape saddle points in later iterations by providing large learning rates at the upper bound.
And once escaping from the saddle points and in a convex region of a minimum, the small learning rates help lock the minimum.
The large learning rate is not supposed to have a strong effect after reaching a convex and locking a minimum as the gradients around a minimum are usually small.

In this work, we would like to try an exponential-range cyclical learning rate, also proposed in \cite{smith_cyclical_2017}.
Given a half-cycle size $N_c$ (that is, it takes $N_c$ iterations for the rate to go from the lower to the upper bound or from the upper to the lower bound), a counter $c$ indicating the current cycle, and $k$ indicating the current iteration, we have
\begin{equation}\label{eq:cyclical-learning-rate}
    \eta(k) = \eta_{low} + \max(0, 1-s)\times(\eta_{high}-\eta_{low})\times\gamma^k
\end{equation}
where $s \equiv \left\lvert \frac{k}{N_c} - 2c + 1\right\rvert$ and $0 \le s \le 1$.
Figure \ref{fig:cyclic-swa-tests-lr-hist} in chapter \ref{chap:pinn-cases} shows an example of how learning rate changes with iterations. 

Another training technique we would like to test together with the cyclical learning rate is the stochastic weight averaging (SWA) \cite{izmailov_averaging_2019}.
SWA basically averages the $\vec{\theta}^k$ from the last few iterations to get the final model parameters:
\begin{equation}
    \vec{\theta} = \frac{1}{N_{SWA}}\sum\nolimits_{i=N_k-N_{SWA}+1}^{N_k} \vec{\theta}^i
\end{equation}
where $N_{SWA}$ denotes the number of iterations to be used in the averaging, and $N_k$ represents the total iterations of the iterative optimization methods.

Izmailov et al. \cite{izmailov_averaging_2019} showed that SWA gave a better accuracy and generalizability of the trained models.
The intuitive mechanism behind SWA is that, on the hypersurface of $r(\Theta)$, the optimizer rarely finds exact minimum but usually circulating in the vicinity of that location.
The last few iterations at the end of training are composed of the coordinates surrounding the actual coordinates of the minimum loss on the hypersurface of $r(\Theta)$ in the $\Theta$ vector space.
Hence, an average of the $\vec{\theta}^k$ may give us the $\vec{\theta}$ of that sits at the center of these $\vec{\theta}^k$.

Another interpretation of SWA is through the combination of the hypersurfaces from different iterations.
Due to the batched training, the hypersurface of $r(\Theta)$ at each iteration is slightly different.
Hence, the $\vec{\theta}^k$ corresponding to the minimum at one iteration does not give the minimal loss and close-to-zero gradients at the next iteration.
SWA helps create a hypersurface with superposition of hypersurfaces at different iterations, and this new hypersurface has a flatter region surrounding the minimum.
This wide and flat region of minimum represents the generalizability of a trained model: how well the model can predict against inputs it never or rarely saw.
And an average of $\vec{\theta}^k$ sits at the center of this wide and flat region of minimum.

As Izmailov et al. proposed to use SWA together with the cyclical learning rate, we would like to include SWA into our benchmarks.

    \subsection{Nonlinear Conjugate-Gradient Optimizer and Line Search}\label{sec:ncg}
    %! TEX root = main.tex

The gradient-descent method and its derived methods all suffer from the problems of choosing right learning rates.
The learning rate scheduling adjusts the value according to pre-defined formulae rather than in an adaptive fashion.
Hence, using a scheduling does not resolve the problem.

Moreover, using negative gradients as the moving directions to find the minimal loss does not mean the overall descent is the fastest.
For example, at $k$-th iteration, the negative gradient $- \nabla_{\Theta} r(\vec{\theta}^k)$ just means the direction that descends the fast at $\Theta=\vec{\theta}^k$.
It does not mean the minimum sits at this direction.
Another real-world example is reaching the lowest-elevation point in a mountain area.
Sometimes it is faster to reach the lowest point by climbing and crossing a mountain rather than always following the downhill directions.

Nonlinear conjugate-gradient (CG) methods provides an alternative searching direction, and a proper line-search algorithm helps determine the learning rate.
In this work, we implemented a variant of CG proposed by Hager and Zhang \cite{hager_new_2005,hager_survey_2006,hager_algorithm_2006}.
We also implemented an inexact line-search algorithm proposed by the same authors.
At $k$-th iteration in CG, the general update formula is
\begin{equation}\label{eq:cg-update-formula}
    \begin{aligned}
        &\vec{d}^k = 
        \left\{
            \begin{array}{ll}
                -\vec{g}^0 & \text{if\ }k = 0 \\
                -\vec{g}^k + \beta_{k-1}\vec{d}^{k-1} & \text{otherwise}
            \end{array}
        \right. \\
        &\vec{\theta}^{k+1} = \vec{\theta}^k + \alpha_k \vec{d}^k
    \end{aligned}
\end{equation}
where $\vec{d}^k$ is the searching direction, and $\alpha_k$ is the step size (or the learning rate).
$\vec{g}^k$ is the gradient at $k$-th iteration.
In our work, this corresponds to $\nabla_{\Theta} r(\Theta=\vec{\theta}^k)$.
The vanilla gradient-descent method can be deemed as a special case of \eqref{eq:cg-update-formula}, in which $\beta_0=\beta_1=\cdots=0$.
Generally speaking, each iteration in CG methods can be seen as starting a new search at a direction between the current search direction and the fastest descending direction at this location.

Different CG methods provide different approaches to evaluate $\beta_k$.
That is, the resulting searching directions from different CG methods are different.
In our work, we used the formula proposed in \cite{hager_new_2005,hager_algorithm_2006}:
\begin{equation}\label{eq:hager-cg-beta}
    \beta_k = \max\left(\hat{\beta}_k, \eta_k\right)
\end{equation}
where
\begin{equation}\label{eq:hager-cg-etak}
    \eta_k = \frac{-1}{\lVert\vec{d}^k\rVert \min \left(\eta, \lVert\vec{g}^k\rVert\right)}
\end{equation}
and
\begin{equation}\label{eq:hager-cg-beta-hat-k}
    \hat{\beta_k} = \left(
        \vec{y}^k -
        2\vec{d}^k
        \frac{\lVert\vec{y}^k\rVert^2}{\left(\vec{d}^k\right)^\mathsf{T}\cdot \vec{y}^k}
    \right)^\mathsf{T}
    \cdot 
    \frac{\vec{g}^{k+1}}{\left(\vec{d}^k\right)^\mathsf{T}\cdot\vec{y}^k}
\end{equation}
The vector $\vec{y}^k$ is defined as $\vec{y}^k=\vec{g}^{k+1}-\vec{g}^k$.
$\eta > 0$ is a user defined constant to control the lower bound of allowed $\beta_k$.
$\hat{\beta_k}$ in \eqref{eq:hager-cg-beta-hat-k} may be negative, meaning we rewind somehow on the current search direction before starting a new search.
By limiting the lower bound, we can limit how much this rewinding effect can be.
A rewinding is not necessarily a bad phenomenon.
It can be seen as a restart on the search when current search has no progress.
We use $\eta=0.01$ for all cases in this work.

As for the learning rate $\alpha_k$, theoretically, $\alpha_k$ is the value that makes $r(\vec{\theta}^k+\alpha_k\vec{d}^k)$ the absolute minimum on the direction $\vec{d}^k$.
A line search algorithm that finds such $\alpha_k$ is called an exact line search algorithm.
However, exact line search is expensive.
Using exact line search hurts the performance advantage of CG (as well as gradient-descent methods).
It is therefore more common to use inexact line search algorithms.
In inexact line search, at $k$-th iteration, we only seek an $\alpha_k$ that helps us land at a new location along $\vec{d}^k$ where
\begin{enumerate}[noitemsep,topsep=-12pt]
    \item we have a sufficient decrease in the loss and
    \item the slope in the $\vec{d}^k$ direction at the new location is milder than that at the beginning of the current iteration.
\end{enumerate}
Several mathematical conditions exit to evaluate these two textual statements quantitatively, including the Wolfe conditions, the strong Wolfe conditions, and the approximate Wolfe conditions.
We ignore the mathematical details and expressions of the two conditions here to help focus on the meaning rather than the formulae.
Please refer to \cite[Chapter~3]{nocedal_numerical_2006} for mathematical details.

Alongside the CG variant \eqref{eq:hager-cg-beta}-\eqref{eq:hager-cg-beta-hat-k}, Hager and Zhang also proposed a companion inexact line search algorithm to find the learning rate $\alpha_k$ that assists this particular GC method to converge.
We have implemented both the CG and this line search in this work.
They will be used in some benchmarks to see if they can be beneficial to the training of data-free PINNs.
% vim:ft=tex

\section{Remarks on Differences and Similarities with Conventional Numerical Methods}\label{sec:pinn-diff}
%! TEX root = main.tex
In this section, we would like to make a brief analogy between traditional numerical methods and PINNs.
When solving differential equations numerically, we can describe the solution workflows of most numerical methods with five stages:
\begin{enumerate}[nolistsep]
    \item Designing the approximate solution with undetermined parameters
    \item Choosing proper approximation for derivatives
    \item Obtaining the modified equations by substituting approximate derivatives into the differential equations, IC, and BCs
    \item Generating a system of linear/nonlinear algebraic equations
    \item Solving the system of equations
\end{enumerate}

For example, to solve $\frac{\diff^2 U(x)}{\diff x^2}=s(x)$, the most naive spectral method \cite{trefethen_spectral_2000} approximates the solution with $U(x)\approx G(x; \Theta)\equiv\sum\limits_{i=1}^{N}\theta_i\phi_i(x)$, where $\theta_i$ represents the free model parameters; $\phi_i(x)$ denotes a set of either polynomials, trigonometric functions, or complex exponentials; and $N$ is the number of terms in the approximation.
Next, approximating the 1st-order derivative $\frac{\diff U(x)}{\diff x}$ is straightforward--we can assume $\frac{\diff U(x)}{\diff x} \approx \pdiff{G(x; \Theta)}{x}=\sum\limits_{i=1}^{N}\theta_i \frac{\diff \phi_i(x)}{\diff x}$.
The 2nd-order derivative may follow the same workflow: $\frac{\diff^2 U(x)}{\diff x^2} \approx \frac{\partial^2 G(x; \Theta)}{\partial x^2}=\sum\limits_{i=1}^{N}\theta_i \frac{\diff^2 \phi_i(x)}{\diff x^2}$.
$\phi_i(x)$ is known, so the derivatives $\frac{\diff \phi_i(x)}{\diff x}$ and $\frac{\diff^2 \phi_i(x)}{\diff x^2}$ are analytical.

Substitute the approximate derivatives into the differential equation, we obtain the residual function in continuous space: $r(x; \Theta) \equiv \sum\limits_{i=1}^{N}\theta_i \frac{\diff^2 \phi_i(x)}{\diff x^2} - s(x)$.

Finally, to determine the actual values of $\theta_i$, one approach is to use $N$ distinct $x$ values at which $r(x; \Theta) = 0$.
This results in a system of linear equations: 
\begin{equation}\label{eq:spectral-linear-sys}
    \begin{bmatrix}
        \frac{\diff^2 \phi_1}{\diff x^2}(x_1) & \cdots & \frac{\diff^2 \phi_N}{\diff x^2}(x_1) \\
        \vdots & \ddots & \vdots \\
        \frac{\diff^2 \phi_1}{\diff x^2}(x_N) & \cdots & \frac{\diff^2 \phi_N}{\diff x^2}(x_N) \\
    \end{bmatrix}
    \begin{bmatrix}
        \theta_1 \\ \vdots \\ \theta_N
    \end{bmatrix}
    - 
    \begin{bmatrix}
        s(x_1) \\ \vdots \\ s(x_N)
    \end{bmatrix}
    = 0
\end{equation}
Solving this linear system determines the values of $\theta_i$ and conclude the solving workflow of this naive spectral method.
The obtained $\theta_i$ guarantees that the residuals are zero at least on the $N$ chosen $x$ coordinates.

Though this example uses a spectral method, the workflow also applies to many other numerical methods, such as finite difference methods, which can be reformatted as a form of spectral method.

Alternatively, some numerical methods solve $\theta_i$ through finding the values that minimize the square of the residual across the whole $x$ domain:
\begin{equation}\label{eq:least-square-fem}
    \begin{aligned}
        \begin{bmatrix}
            \theta_1 \\ \vdots \\ \theta_N
        \end{bmatrix}
        & =
        \operatorname*{arg\,min}\limits_{\theta_i}
        \int\limits_{x}\left[r(x; \Theta)\right]^2\diff x \\
        & =
        \operatorname*{arg\,min}\limits_{\theta_i}
        \int\limits_{x}\left[\sum\limits_{i=1}^{N}\theta_i \frac{\diff^2 \phi_i(x)}{\diff x^2} - s(x)\right]^2\diff x
    \end{aligned}
\end{equation}
However, in these numerical methods, the optimization is done by solving the zero-slope conditions directly:
\begin{equation}
    \pdiff{\left[r(x; \Theta)\right]^2}{\theta_1} = 
    \pdiff{\left[r(x; \Theta)\right]^2}{\theta_2} = 
    \cdots =
    \pdiff{\left[r(x; \Theta)\right]^2}{\theta_N} = 
    0
\end{equation}
which also results in a linear system:
\begin{equation}\label{eq:spetral-least-equare}
    \begin{bmatrix}
        \int\limits_{x}
        \frac{\diff^2 \phi_1}{\diff x^2}
        \frac{\diff^2 \phi_1}{\diff x^2}
        \diff x
        &
        \cdots
        &
        \int\limits_{x}
        \frac{\diff^2 \phi_N}{\diff x^2}
        \frac{\diff^2 \phi_1}{\diff x^2}
        \diff x \\
        \vdots & \ddots & \vdots \\
        \int\limits_{x}
        \frac{\diff^2 \phi_1}{\diff x^2}
        \frac{\diff^2 \phi_N}{\diff x^2}
        \diff x
        &
        \cdots
        &
        \int\limits_{x}
        \frac{\diff^2 \phi_N}{\diff x^2}
        \frac{\diff^2 \phi_N}{\diff x^2}
        \diff x \\
    \end{bmatrix}
    \begin{bmatrix}
        \theta_1 \\ \vdots \\ \theta_N
    \end{bmatrix}
    - 
    \begin{bmatrix}
        \int\limits_{x}s(x)\frac{\diff \phi_1}{\diff x}(x) \diff x \\
        \vdots \\
        \int\limits_{x}s(x)\frac{\diff \phi_1}{\diff x}(x) \diff x
    \end{bmatrix}
    = 0
\end{equation}
As $\phi_i$ is given, the integrals can be evaluated analytically.
Moreover, a proper choice of $\phi_i$ can make the coefficient matrix in \eqref{eq:spetral-least-equare} sparse, making it cheap and fast to solve.
Finally, $\Theta=\{\theta_1,\cdots\theta_N\}$ is determined.
In this approach, the obtained $\theta_i$ guarantees that the integrated residual over the whole domain is minimal, but it does not guarantee a zero residual unless $r(x; \Theta) = \sum\limits_{i=1}^{N}\theta_i \frac{\diff^2 \phi_i(x)}{\diff x^2} - s(x)$ has zero roots with respect to $\Theta$ for any given $x$.

If we replace the integral in \eqref{eq:least-square-fem} with a Monte-Carlo numerical integration:
\begin{equation}\label{eq:least-square-monte-carlo}
    \begin{aligned}
        \begin{bmatrix}
            \theta_1 \\ \vdots \\ \theta_N
        \end{bmatrix}
        & =
        \operatorname*{arg\,min}\limits_{\theta_i}
        \int\limits_{x}\left[r(x; \Theta)\right]^2\diff x \\
        &\approx
        \operatorname*{arg\,min}\limits_{\theta_i}
        \frac{L_x}{N}\sum\limits_{m=1}^{N}\left[r(x_m; \Theta)\right]^2 \\
        & =
        \operatorname*{arg\,min}\limits_{\theta_i}
        \sum\limits_{m=1}^{N}\left[\sum\limits_{i=1}^{N}\theta_i \frac{\diff^2 \phi_i(x_m)}{\diff x^2} - s(x_m)\right]^2
    \end{aligned}
\end{equation}
where $L_x$ is the computational domain, and $N$ is the number of random points sampled from the computational domain.
We can see that \eqref{eq:least-square-monte-carlo} is actually the loss function in PINNs (equations \eqref{eq:residual-norms} and \eqref{eq:total-residual}) tailored to this specific example.
(The BCs are omitted from the beginning of this example for simplicity.)
Equations \eqref{eq:least-square-fem}-\eqref{eq:spetral-least-equare} are called least-square finite-element methods (least-square FEMs).
This shows PINNs can be seen as a type of global least-square FEMs.

With the spectral and least-square finite element methods in mind, we can see the analogy between PINNs and conventional numerical methods.
Especially when comparing to the least-square FEMs, their workflows are very close.
The main difference is that PINNs replace the analytical integration with a Monte-Carlo numerical integration.
Further, while the least-square FEMs solve zero-slope conditions to find the minimum residuals, PINNs rely on searching the minimal residuals on a complicated hypersurface of residuals iteratively.
It is due to the complexity and the nonlinearity of the models in PINNs, where neither solving zero-residual conditions on selected points nor solving zero-slope conditions can reduce problems to a linear system. 

The second difference is how to approximate derivatives.
Conventional numerical methods use analytical or numerical differentiation of the models to approximate derivatives.
On the other hand, modern PINNs depend on automatic differentiation due to the complexity and nonlinearity in the networks.

The two differences play an important role in solution accuracy and time-to-solution.
Solving linear systems is intuitively faster than solving a nonlinear optimization problems through 1st-order optimization methods and is cheaper than 2nd-order methods in terms of memory usage.
And while automatic differentiation is powerful, it requires a nontrivial computer memory.
As seen in section \ref{sec:ad}, automatic differentiation needs to memorize all computational graphs for derivatives.
And high-order derivatives needs also to memorize the computational graphs of lower-order derivatives, making the computational graph to grow exponentially.

% vim:ft=tex


\section{Code Implementation}\label{sec:pinn-code-impl}
%! TEX root = main.tex
The PINN solvers used in the benchmarks in chapter~\ref{chap:pinn-cases} were implemented with the help of NVIDIA's Modulus \cite{noauthor_modulus_nodate}.
Modulus utilizes symbolic expression, constructive solid geometry, computational graphs, and PyTorch to make defining a PINN PDE solver easier.
It also provides some pre-defined PDE solvers and neural network architectures.
Regular users only need to configure the computational domains and constraint instances to evaluate each loss term in equation \eqref{eq:total-residual-weighted}.
(For example, the number of training points per batch, sampling regions, or weights.)
These configurations are done with a combination of YAML files and Python code.
For simple use cases, the Python code only consists of several lines defining how to interpret the key-value pairs in the YAML files and calling APIs to set up loss terms.

Another feature of Modulus is using SymPy's symbolic expression \cite{meurer_sympy_2017} as a user interface.
For example, if users want to sample points from the region where $x > 0.5$, they add a SymPy symbolic expression instance of \lstinline{sympy.Gt(x, 0.5)} to the corresponding constraint instance.
Internally, Modulus does not rely on symbolic calculation.
Instead, it translates users' symbolic expressions to numeric calculations in PyTorch.
With this symbolic user interface, users are even allowed to define new PDEs by using symbolic expressions.
Please refer to Modulus' manual for details.

Modulus is still an experimental product at the time of writing.
It lacks documentation for code developers. 
To meet our needs for conducting benchmarks, after reading and understanding the source code of Modulus, we were able to customize and added new code that work with Modulus.
The additions and modifications associated with the present work include:
\begin{enumerate}[nolistsep]
    \item A reimplementation of the unsteady incompressible Navier-Stokes equations.
        Modulus also ships a similar class for the compressible Navier-Stokes equations, which falls back to the same incompressible formulation when the density is a constant.
        We reimplemented it for better control over the source code.
    \item Other non-governing PDEs, such as vorticity, Q criterion, and Neumann and convective boundary conditions.
        These PDEs are subclasses of \lstinline{modulus.pdes.PDES}.
        The first two PDEs are for post-processing, and the remaining two are for evaluating the BC losses.
    \item A binding to PyTorch's \lstinline{CylicLR} learning schedule.
    \item Nonlinear CG and inexact line search as a subclass of \lstinline{torch.optim.Optimizer}.
    \item Reimplementation of the annealing loss aggregation based on equation \eqref{eq:annealing-in-this-work}.
        The original annealing loss aggregation implementation was hard-coded for single-variable PDEs, which does not work with PDE systems like the Navier-Stokes equations.
    \item A solution workflow controller that allows the use of annealing loss aggregation, Adam, nonlinear CG, and SWA in an arbitrary combination. 
    \item An inferencer that allows to output only the desired portion of a PINN model every given iterations. 
    \item Boundary and interior constraint classes that are aware of the counter of training iterations so that they keep re-using the same batch of training points when the iteration counter remains the same.
    They are used when the CG optimizer and the annealing loss aggregation are both enabled.
\end{enumerate}

The code for the mentioned new/reimplemented components is available at \cite{chuang_dissertation_nodate}.
The same reference also contains all other code, scripts, and the container definition files for reproducing the results and figures in the later benchmarks.
% vim:ft=tex

% vim:ft=tex:
