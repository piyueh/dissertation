%! TEX root = main.tex
PetIBM is a C++ library that helps create new incompressible flow solvers for large-scale and distributed GPU computing.
The core design philosophy of PetIBM is to allow the easy creation of a solver that fits into Perot's framework \cite{perot_analysis_1993} for projection methods.
Combining Perot's framework and staggered grids, PetIBM does not need pressure boundary conditions.
Pressure boundary conditions have caused many headaches for CFD practitioners \cite{gresho_pressure_1987,sani_resume_1994}.

On top of the vanilla projection method from Perot, the primary use case of PetIBM is to create solvers of the immersed-boundary projection method (IBPM \cite{taira_immersed_2007}) and its derived works.
With PetIBM, users focus their time on defining components and algorithms in Perot's framework and leave the parallelization to PetIBM.
Solvers created with the PetIBM library run on distributed-memory architectures, including GPU clusters.

To our knowledge, another work similar to PetIBM is IBAMR \cite{griffith_adaptive_2007,bhalla_unified_2013}. 
PetIBM and IBAMR use different immersed-boundary schemes.
IBAMR is based on Peskin's immersed-boundary formulation \cite{Peskin2002} and implements adaptive mesh refinement.
However, IBAMR does not have GPU capability, as far as we know.

Though PetIBM's primary role is a library, it includes several flow solvers for different flow applications and solving schemes.
These solvers serve as examples for users to learn how to create their own.
We also use these solvers to conduct studies in fluid dynamics (e.g., \cite{mesnard_reproducible_2017}).
All baseline data in the PINNs' benchmarks in chapter \ref{chap:pinn-cases} were generated by a decoupled IBPM (decoupled immersed-boundary projection method \cite{li_efficient_2016}) shipped with PetIBM.
In the remaining dissertation, PetIBM often refers to this decoupled IBPM solver rather than the library itself.

The following section briefly introduces the decoupled IBPM.
Section \ref{sec:petibm-impl} describes the higher-level software design of the PetIBM library and the solver.
The decoupled IBPM solver will be used to generate baseline data for benchmarking PINNs, so verification, validation, and performance benchmarks are done and shown in sections \ref{sec:petibm-vv} and \ref{sec:petibm-perf}.

For interested readers, a complete PetIBM installation also includes a basic Navier-Stokes equation solver (for flows without objects), the original IBPM solver, and a solver for moving bodies with prescribed motions.
These solvers, however, are out of this work's scope.
Their descriptions can be found in the documentation of reference \cite{chuang_petibm:_2018}.
% vim:ft=tex
