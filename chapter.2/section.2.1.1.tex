%! TEX root = main.tex
PetIBM solves the Navier-Stokes equations on an extended discretization grid that includes the interior of the immersed boundary.
To model the presence of the boundary, a forcing term is added to the momentum equation and an additional equation for the no-slip condition completes the system.
Variants of the immersed-boundary method (IBM) depend on how one models the forcing.
In PetIBM, we use regularized-delta functions to transfer data between the Eulerian grid and the Lagrangian boundary points.
The system of equations is:

\begin{equation}
\begin{cases}
\frac{\partial \symbf{u}}{\partial t} + \symbf{u} \cdot \nabla \symbf{u} = -\nabla p + \frac{1}{Re} \nabla^2 \symbf{u} + \int_{s}{\symbf{f} \left( \symbf{\xi} \left( \symit{s}, \symit{t} \right) \right) \delta \left( \symbf{\xi} - \symbf{x} \right)} d\symit{s} \\
\nabla \cdot \symbf{u} = 0 \\
\symbf{u} \left( \symbf{\xi} \left( \symit{s}, t \right) \right) = \int_{\symbf{x}}{\symbf{u} \left( \symbf{x} \right)} \delta \left( \symbf{x} - \symbf{\xi} \right) d\symbf{x}
\end{cases}
\end{equation}

where $\symbf{u}$ is the velocity field, $p$ is the fluid pressure, and $Re$ is the Reynolds number.

Currently, PetIBM provides two application codes implementing different versions of the IBM: (1) an immersed-boundary projection method (IBPM) based on the work of \cite{taira_colonius_2007} and (2) a decoupled version of the IBPM proposed by \cite{li_et_al_2016}.
Those two methods fit into the framework of the projection approach of \cite{perot_1993}.
The equations are fully discretized (space and time) to form an algebraic system to be solved for the velocity $u^{n+1}$, the pressure field $\phi$, and the Lagrangian forces $\tilde{f}$.
The discretized system is:

\begin{equation}
\left[
\begin{matrix}
A & G & H \\
D & 0 & 0 \\
E & 0 & 0
\end{matrix}
\right]
\left(
\begin{matrix}
u^{n+1} \\
\phi \\
\tilde{f}
\end{matrix}
\right)
=
\left(
\begin{matrix}
r^n \\
0 \\
u_B^{n+1}
\end{matrix}
\right)
+
\left(
\begin{matrix}
bc_1 \\
bc_2 \\
0
\end{matrix}
\right)
\end{equation}

where $D$, $G$, and $A$ are the divergence, gradient, and implicit operators, respectively.
$E$ and $H$ are the interpolation and spreading operators, respectively, used to transfer the data between the Eulerian grid and the Lagrangian boundary points.
On the right-hand side, $r^n$ gathers all the explicit terms and $u_B^{n+1}$ is the known (prescribed) boundary velocity; $bc_1$ and $bc_2$ contain the boundary terms that arise from the discretization of momentum and continuity equations, respectively.

In the IBPM, we solve a modified Poisson system for the pressure field and Lagrangian forces, coupled together.
This way, the divergence-free condition and no-slip constraint are simultaneously enforced on the velocity field at the end of the time step.
The fully discretized system can be cast into the following:

\begin{equation}
\left[
\begin{matrix}
A & Q_2 \\
Q_1 & 0
\end{matrix}
\right]
\left(
\begin{matrix}
u^{n+1} \\
\lambda
\end{matrix}
\right)
=
\left(
\begin{matrix}
r_1 \\
r_2
\end{matrix}
\right)
\end{equation}

with

\begin{equation*}
Q_1 \equiv \left[ \begin{matrix} D \\ E \end{matrix} \right] ;\
Q_2 \equiv \left[ G, H \right] ;\
\lambda \equiv \left( \begin{matrix} \phi \\ \tilde{f} \end{matrix} \right) ;\
r_1 \equiv r^n + bc_1 ;\
r_2 \equiv \left( \begin{matrix} bc_2 \\ u_B^{n+1} \end{matrix} \right)
\end{equation*}

In practice, we never form the full system.
Instead, we apply a block-LU decomposition as follows:

\begin{equation}
\left[
\begin{matrix}
A & 0 \\
Q_1 & -Q_1A^{-1}Q_2
\end{matrix}
\right]
\left[
\begin{matrix}
I & A^{-1}Q_2 \\
0 & I
\end{matrix}
\right]
\left(
\begin{matrix}
u^{n+1} \\
\lambda
\end{matrix}
\right)
=
\left[
\begin{matrix}
A & 0 \\
Q_1 & -Q_1A^{-1}Q_2
\end{matrix}
\right]
\left(
\begin{matrix}
u^* \\
\lambda
\end{matrix}
\right)
=
\left(
\begin{matrix}
r_1 \\
r_2
\end{matrix}
\right)
\end{equation}

Thus, we retrieve the sequence of operations of the traditional projection method.
We solve a system for an intermediate velocity field that is corrected, after solving a modified Poisson system for the variable $\lambda$, to enforce the divergence-free condition and the no-slip constraint at the location of the immersed boundary.
The sequence is:

\begin{align}
& A u^* = r_1 \\
& Q_1A^{-1}Q_2 \lambda = Q_1 u^* - r_2 \\
& u^{n+1} = u^* - A^{-1}Q_1 \lambda
\end{align}

The IBPM implemented in PetIBM solves, at every time step, Equations (5) to (6).
(Note: the inverse of the implicit operator $A^{-1}$ is approximated by a finite Taylor series expansion.)


The IBPM requires solving, at each time step, an expensive modified Poisson system, $Q_1A^{-1}Q_2$, whose non-zero structure changes when the location of the immersed boundary is moving.
In the PetIBM implementation of the decoupled IBPM, we apply a second block-LU decomposition to decouple the pressure field from the Lagrangian forces and recover a classical Poisson system.
The fully discretized algebraic system can be cast into:

\begin{equation}
\left[
\begin{matrix}
A & H & G \\
E & 0 & 0 \\
D & 0 & 0
\end{matrix}
\right]
\left(
\begin{matrix}
u^{n+1} \\
\tilde{f} \\
\phi
\end{matrix}
\right)
=
\left(
\begin{matrix}
r^n \\
u_B^{n+1} \\
0
\end{matrix}
\right)
+
\left(
\begin{matrix}
bc_1 \\
0 \\
bc_2
\end{matrix}
\right)
\end{equation}

The velocity $u^{n+1}$ and the Lagrangian forces $\tilde{f}$ are coupled together to form a new unknown $\gamma^{n+1}$, as follows:

\begin{equation}
\left[
\begin{matrix}
\bar{A} & \bar{G} \\
\bar{D} & 0
\end{matrix}
\right]
\left(
\begin{matrix}
\gamma^{n+1} \\
\phi
\end{matrix}
\right)
=
\left(
\begin{matrix}
\bar{r_1} \\
\bar{r_2}
\end{matrix}
\right)
\end{equation}

where

\begin{equation*}
\bar{A} \equiv \left[ \begin{matrix} A & H \\ E & 0 \end{matrix} \right] ;\
\bar{G} \equiv \left[ \begin{matrix} G \\ 0 \end{matrix} \right] ;\
\bar{D} \equiv \left[ \begin{matrix} D & 0 \end{matrix} \right]
\end{equation*}

and

\begin{equation*}
\gamma^{n+1} \equiv \left( \begin{matrix} u^{n+1} \\ \tilde{f} \end{matrix}\right) ;\
\bar{r_1} \equiv \left( \begin{matrix} r_n + bc_1 \\ u_B^{n+1} \end{matrix}\right) ;\
\bar{r_2} \equiv bc_2
\end{equation*}

Two successive block-LU decompositions are applied to decouple the Lagrangian forces $\tilde{f}$ from $\gamma^{n+1}$ and to decouple the velocity from the pressure field.

The first block-LU decomposition decouples the pressure field from the new unknown $\gamma^{n+1}$:

\begin{equation}
\left[
\begin{matrix}
\bar{A} & 0 \\
\bar{D} & -\bar{D}\bar{A}^{-1}\bar{G}
\end{matrix}
\right]
\left[
\begin{matrix}
I & \bar{A}^{-1}\bar{G} \\
0 & I
\end{matrix}
\right]
\left(
\begin{matrix}
\gamma^{n+1} \\
\phi
\end{matrix}
\right)
=
\left[
\begin{matrix}
\bar{A} & 0 \\
\bar{D} & -\bar{D}\bar{A}^{-1}\bar{G}
\end{matrix}
\right]
\left(
\begin{matrix}
\gamma^* \\
\phi
\end{matrix}
\right)
=
\left(
\begin{matrix}
\bar{r_1} \\
\bar{r_2}
\end{matrix}
\right)
\end{equation}

which leads to the following sequence of operations:

\begin{align}
& \bar{A} \gamma^* = \bar{r_1} \\
& \bar{D}\bar{A}^{-1}\bar{G} \phi = \bar{D} \gamma^* - \bar{r_2} \\
& \gamma^{n+1} = \gamma^* - \bar{A}^{-1}\bar{G} \phi
\end{align}

A second block-LU decomposition is applied to the first equation above:

\begin{equation}
\left[
\begin{matrix}
A & 0 \\
E & -EA^{-1}H
\end{matrix}
\right]
\left[
\begin{matrix}
I & A^{-1}H \\
0 & I
\end{matrix}
\right]
\left(
\begin{matrix}
u^* \\
\tilde{f}
\end{matrix}
\right)
=
\left[
\begin{matrix}
A & 0 \\
E & -EA^{-1}H
\end{matrix}
\right]
\left(
\begin{matrix}
u^{* *} \\
\tilde{f}
\end{matrix}
\right)
=
\left(
\begin{matrix}
r^n + bc_1 \\
u_B^{n+1}
\end{matrix}
\right)
\end{equation}

and we end up with the following sequence:

\begin{align}
& A u^{* *} = r^n + bc_1 \\
& EA^{-1}H \tilde{f} = E u^{* *} - u_B^{n+1} \\
& u^* = u^{* *} - A^{-1}H \tilde{f}
\end{align}

The decoupled version of the IBPM implemented in PetIBM solves, at every time step, Equations (15) to (17) followed by Equations (12) and (13).
% vim:ft=tex