%! TEX root = main.tex

Immersed-boundary methods solve fluid flow problems with the presence of solids by adding a body-force term to the Navier-Stokes equations.
This body-force term mimics the interface between fluid and solids.
Depending on the chosen formulations and schemes, the solids may be either rigid or deformable.
They can be stationary objects, objects moving under prescribed motions, or objects moving under feedback forces from the fluid.
Different immersed-boundary methods may use different formulations for this body-force term to model the fluid-solid interfaces.

To be more precise, this forcing term approximates the force applied to the fluid by solids.
As this forcing term is part of the fluid governing equations defined in Eulerian coordinate systems, it is resolved on the same computational grid of the fluid, and solid-fluid interfaces are not boundaries of the fluid's computational domain anymore.
In other words, body-fitting meshes are unnecessary.
A benefit of such an approach is that a structured grid can be applied to solve flow problems with the presence of an object with complex geometry.

Consider a fluid problem with a fluid-solid interface.
$\Omega$ and $\Gamma$ correspond to the whole domain of our interest and the fluid-solid interface.
The fluid-solid interface may be moving when viewed from a fixed coordinate system.
Given a time $t$ and a spatial coordinate $\vec{x}$ in an Eulerian coordinate system defined in $\Omega$, we can write the following Navier-Stokes equations for incompressible flow and the immersed-boundary projection method (IBPM):
\begin{subequations}\label{eq:mod-ns}
    \begin{align}[left=\empheqlbrace\,]
        &\pdiff{\vec{u}}{t} + \left(\vec{u} \cdot \nabla\right) \vec{u}
            =
            -\frac{1}{\rho}\nabla p + \frac{1}{Re} \nabla^2 \vec{u} + 
            \int\limits_{\Gamma} \vec{f}\left(\vec{\xi}\left(s, t \right), t\right) \delta\left(\vec{\xi}\left(s, t\right)-\vec{x}\right)\diff {\Gamma} 
            \label{eq:mod-ns-momentum} \\
        &\nabla \cdot \vec{u} = 0 \label{eq:mod-ns-cont}\\
        &\int\limits_{\Omega} \vec{u}\left(\vec{x}, t\right) \delta \left(\vec{x}-\vec{\xi}\left(s, t\right)\right)\diff\Omega
            =
            \vec{u}_{\Gamma}\left(s, t\right)
            \label{eq:mod-ns-body-vel}
    \end{align}
\end{subequations}

$\vec{u}$, $p$, $Re$, and $\rho$ are the velocity vector, pressure, Reynolds number, and fluid density.
$\vec{u}$ and $p$ are functions of $\vec{x}$ and $t$.
Though $\rho$ and viscosity can also be functions of $\vec{x}$ and $t$ for incompressible flow, we simply treat them as a constant in this work.
$s$ denotes the coordinate in a Lagrangian coordinate system defined on $\Gamma$.
For example, such a Lagrangian coordinate system is a curve in 2D problems, and a single scalar $s$ is enough to represent a coordinate on the interface.
$\vec{\xi}(s, t)$ is the coordinate in the Eulerian coordinate system that correspond to a Lagrangian coordinate $s$ on $\Gamma$ and at time $t$.
$\vec{u}_\Gamma$ represents the velocity of the fluid-solid interface.
Note $\vec{u}_\Gamma$ is a concept from the viewpoint of the Lagrangian coordinate system.
$\vec{f}$ is the vector of force applied on fluid due to the presence of the fluid-solid interface.
It is defined in $\Omega$ and the Eulerian coordinate system, so it is a function of $\vec{x}$ and $t$.

$\delta$ is the Dirac delta function, i.e., $\delta(\vec{\xi}-\vec{x})$ equals $1$ when $\vec{\xi} = \vec{x}$ and equals $0$ otherwise.
With this Dirac delta function, the last term in \eqref{eq:mod-ns-momentum} is non-zero only at $\vec{x}$ corresponding to the fluid-solid interface.
In other words, one can also write separate momentum equations without using the Dirac function:
\begin{equation}
\left\{
    \begin{array}{ll}
        \pdiff{\vec{u}}{t} + \left(\vec{u} \cdot \nabla\right) \vec{u}
            = -\frac{1}{\rho}\nabla p + \frac{1}{Re} \nabla^2 \vec{u}\text{,} & \vec{x} \notin \Gamma \\
        \pdiff{\vec{u}}{t} + \left(\vec{u} \cdot \nabla\right) \vec{u}
            = -\frac{1}{\rho}\nabla p + \frac{1}{Re} \nabla^2 \vec{u} + \vec{f}\left(\vec{x}, t \right)\text{,} & \vec{x} \in \Gamma \\
    \end{array}
\right. 
\end{equation}
It is clear now that only $\vec{x}$ exactly on fluid-solid interfaces is under the influence of the added body force.

The same logic applies to the integral in \eqref{eq:mod-ns-body-vel}.
In continuous space, equation \eqref{eq:mod-ns-body-vel} simply means $\vec{u}\left(\vec{x}, t\right)=\vec{u}_\Gamma\left(s, t\right)$ for $\vec{x} \in \Gamma$ and when $\vec{x}=\vec{\xi}\left(s, t\right)$.
It means the flow velocity on $\Gamma$ must satisfy the no-slip condition.
As $\Gamma$ is not a boundary of $\Omega$, this no-slip condition can not be specified as a boundary condition and hence is included in \eqref{eq:mod-ns}.

Though the integrals in the last term of \eqref{eq:mod-ns-momentum} and \eqref{eq:mod-ns-body-vel} seem unnecessary, having the integral or not makes a difference when applying discretization in numerical methods.
Because we are not using body-fitting mesh, and the fluid-solid interface is arbitrary, it is almost impossible to have computational nodes located exactly on $\Gamma$.
Hence, to capture the effect of the interface, the delta function needs to be relaxed, i.e., $\delta \ne 0$ for $\lVert \vec{\xi}-\vec{x} \rVert$ smaller than a certain predefined tolerance.
Alternatively, it can be made into some smooth function that approaches zero when $\lVert \vec{\xi}-\vec{x} \rVert$ approaches infinity.
This kind of delta function is called regularized Dirac delta function.
Using a regularized delta function makes the fluid-solid interface diffuse: if we plot the integral in the last term of equation \eqref{eq:mod-ns-momentum} on a discretized 2D Cartesian grid of $\Omega$, the $\Gamma$ will look like a band rather than a curve.

In PetIBM, we use the following regularized Dirac function:
\begin{equation}\label{eq:regularized-dirac}
    \delta\left(r\right)=\left\{
        \begin{array}{ll}
            \frac{1}{6 \Delta r}\left[
                5-3 \frac{r}{\Delta r}-\sqrt{-3\left(1-\frac{r}{\Delta r}\right)^{2}+1}
            \right] & \text { for } 0.5 \Delta r \le r \le 1.5 \Delta r \\
            \frac{1}{3 \Delta r}\left[
                1+\sqrt{-3\left(\frac{r}{\Delta r}\right)^{2}+1}
            \right] & \text { for }r \le 0.5 \Delta r \\
            0 & \text { otherwise }
        \end{array}
    \right.
\end{equation}
where $r \equiv \lVert\vec{\xi}-\vec{x}\rVert$, and $\Delta r$ is a user-provided parameter to determine the effective range of the regularized Dirac function.

The next step is to discretize the governing equations.
The IBPM is based on the following block-matrix system that can be generated from most discretization and time marching schemes:
\begin{equation}\label{eq:block-sys-ns}
    \begin{bmatrix}
        A & G & H \\
        D & 0 & 0 \\
        E & 0 & 0
    \end{bmatrix}
    \begin{bmatrix}
        \vec{u}^{n+1} \\
        \vec{p}^{n+1} \\
        \vec{f}^{n+1}
    \end{bmatrix}
    =
    \begin{bmatrix}
        \vec{\gamma}^n \\
        0 \\
        \vec{u}_\Gamma^{n+1}
    \end{bmatrix}
    +
    \begin{bmatrix}
        \vec{bc}_1 \\
        \vec{bc}_2 \\
        0
    \end{bmatrix}
\end{equation}
where $A$, $G$, and $D$ correspond to the $\frac{1}{\Delta t}I+\alpha\frac{1}{Re}L$, gradient, and divergence operators of a given spatial discretization. 
$L$ is the Laplacian operator, and $\alpha$ is a coefficient that depends on the chosen time marching scheme for the diffusion term.
$H$ and $E$ are the operators from substituting equation \eqref{eq:regularized-dirac} into equation \eqref{eq:mod-ns} and discretizing the forcing terms and \eqref{eq:mod-ns-body-vel}.
Only matrix $A$ is a square matrix, while the others are not.
Superscript $n+1$ and $n$ denote the $n+1$-th and $n$-th time steps.
$\vec{\gamma}^n$ represents the results of explicit terms from the selected time marching schemes.
Though having a superscript $n$, $\vec{\gamma}^n$ can also depend on solutions from $n-1$, $n-2$, and so on, based on the chosen temporal schemes.
$\vec{bc}_1$ and $\vec{bc}_2$ are the artifacts generated from eliminating BCs from operators.

Reference \cite{taira_immersed_2007} shows an example of how these operators are defined using a staggered Cartesian grid for spatial discretization, Adams-Bashforth for marching the convection term, and Crank-Nicolson for marching the diffusion term.
Here, we assume we already have these operators to save space.
We can focus on the solution scheme of this block matrix system using the decoupled IBPM in reference \cite{li_efficient_2016}.

Equation \ref{eq:block-sys-ns} is indefinite and difficult to solve.
To solve such a system, in decoupled IBPM, we define the following composite matrices and composite vectors: 
\begin{equation}\label{eq:composites}
    \begin{gathered}
        \bar{A} \equiv \begin{bmatrix} A & H \\ E & 0 \end{bmatrix};\
        \bar{G} \equiv \begin{bmatrix} G \\ 0 \end{bmatrix};\
        \bar{D} \equiv \begin{bmatrix} D & 0 \end{bmatrix}
        \\
        \vec{q}^{n+1} \equiv \begin{bmatrix} \vec{u}^{n+1} \\ \vec{f}^{n+1} \end{bmatrix};\
        \bar{\vec{\gamma}}_1 \equiv \begin{bmatrix} \vec{\gamma}^n + \vec{bc}_1 \\ \vec{u}_\Gamma^{n+1}\end{bmatrix};\
        \bar{\vec{\gamma}}_2 \equiv \vec{bc}_2
    \end{gathered}
\end{equation}
The original block system in \eqref{eq:block-sys-ns} can be re-arranged to:
\begin{equation}
    \begin{bmatrix}
        \bar{A} & \bar{G} \\
        \bar{D} & 0
    \end{bmatrix}
    \begin{bmatrix}
        \vec{q}^{n+1} \\
        \vec{p}^{n+1}
    \end{bmatrix}
    =
    \begin{bmatrix}
        \bar{\vec{\gamma}}_1 \\
        \bar{\vec{\gamma}}_2
    \end{bmatrix}
\end{equation}

The next step is to apply two successive block-LU decompositions.
The first block-LU decomposition decouples the pressure field from the new unknown $\vec{q}^{n+1}$:
\begin{equation}
    \begin{bmatrix}
        \bar{A} & 0 \\
        \bar{D} & -\bar{D}\bar{A}^{-1}\bar{G}
    \end{bmatrix}
    \begin{bmatrix}
        I & \bar{A}^{-1}\bar{G} \\
        0 & I
    \end{bmatrix}
    \begin{bmatrix}
        \vec{q}^{n+1} \\
        \vec{p}^{n+1}
    \end{bmatrix}
    =
    \begin{bmatrix}
        \bar{A} & 0 \\
        \bar{D} & -\bar{D}\bar{A}^{-1}\bar{G}
    \end{bmatrix}
    \begin{bmatrix}
        \vec{q}^* \\
        \vec{p}^{n+1}
    \end{bmatrix}
    =
    \begin{bmatrix}
        \bar{\vec{\gamma}}_1 \\
        \bar{\vec{\gamma}}_2
    \end{bmatrix}
\end{equation}
which leads to the following sequence of operations:
\begin{subequations}
    \begin{align}
        & \bar{A} \vec{q}^* = \bar{\vec{\gamma}}_1\label{eq:dibpm-lu1-1} \\
        & \bar{D}\bar{A}^{-1}\bar{G} \vec{p}^{n+1} = \bar{D} \vec{q}^* - \bar{\vec{\gamma}}_2\label{eq:dibpm-lu1-2} \\
        & \vec{q}^{n+1} = \vec{q}^* - \bar{A}^{-1}\bar{G} \vec{p}^{n+1}\label{eq:dibpm-lu1-3}
    \end{align}
\end{subequations}
Noticing the zero submatrices in $\bar{G}$ and $\bar{D}$, we can split equations \eqref{eq:dibpm-lu1-2} and \eqref{eq:dibpm-lu1-3} to:
\begin{subequations}
    \begin{align}
        & D\bar{A}_{00}^{-1}G \vec{p}^{n+1} = D \vec{u} - \vec{bc}_2\label{eq:dibpm-lu1-4} \\
        & \vec{u}^{n+1} = \vec{u}^* - \bar{A}_{00}^{-1}G \vec{p}^{n+1}\label{eq:dibpm-lu1-5} \\
        & \vec{f}^{n+1} = \vec{f}^* - \bar{A}_{10}^{-1}G \vec{p}^{n+1}\label{eq:dibpm-lu1-6}
    \end{align}
\end{subequations}
where $\bar{A}_{00}^{-1}$ and $\bar{A}_{10}^{-1}$ are the upper-left and lower-left submatrices in $\bar{A}^{-1}$.
The composite operator $\bar{A}$ is still indefinite, and step \eqref{eq:dibpm-lu1-1} is also tough to solve directly.
Therefore, a second block-LU decomposition is applied to solve $\vec{q}^{*}$ in \eqref{eq:dibpm-lu1-1}:
\begin{equation}
    \begin{bmatrix}
        A & 0 \\
        E & -EA^{-1}H
    \end{bmatrix}
    \begin{bmatrix}
        I & A^{-1}H \\
        0 & I
    \end{bmatrix}
    \begin{bmatrix}
        \vec{u}^* \\
        \vec{f}^*
    \end{bmatrix}
    =
    \begin{bmatrix}
        A & 0 \\
        E & -EA^{-1}H
    \end{bmatrix}
    \begin{bmatrix}
        \vec{u}^{* *} \\
        \vec{f}^*
    \end{bmatrix}
    =
    \begin{bmatrix}
        \vec{\gamma}^n + \vec{bc}_1 \\
        \vec{u}_\Gamma^{n+1}
    \end{bmatrix}
\end{equation}
and we end up with the following sequence:
\begin{subequations}
    \begin{align}
        & A u^{* *} = \vec{\gamma}^n + \vec{bc}_1\label{eq:dibpm-lu2-1} \\
        & EA^{-1}H \vec{f}^* = E \vec{u}^{* *} - \vec{u}_\Gamma^{n+1}\label{eq:dibpm-lu2-2} \\
        & \vec{u}^* = \vec{u}^{* *} - A^{-1}H \vec{f}^*\label{eq:dibpm-lu2-3}
    \end{align}
\end{subequations}
The order of the solution process is then:
\begin{subequations}
    \begin{align}
        & A u^{* *} = \vec{\gamma}^n + \vec{bc}_1\label{eq:dibpm-order1-1} \\
        & EA^{-1}H \vec{f}^* = E \vec{u}^{* *} - \vec{u}_\Gamma^{n+1}\label{eq:dibpm-order1-2} \\
        & \vec{u}^* = \vec{u}^{* *} - A^{-1}H \vec{f}^*\label{eq:dibpm-order1-3}\\
        & D\bar{A}_{00}^{-1}G \vec{p}^{n+1} = D \vec{u}^* - \vec{bc}_2\label{eq:dibpm-order1-4} \\
        & \vec{u}^{n+1} = \vec{u}^* - \bar{A}_{00}^{-1}G \vec{p}^{n+1}\label{eq:dibpm-order1-5} \\
        & \vec{f}^{n+1} = \vec{f}^* - \bar{A}_{10}^{-1}G \vec{p}^{n+1}\label{eq:dibpm-order1-6}
    \end{align}
\end{subequations}

Note that because $A$ has a format of $\frac{1}{\Delta t}I + \alpha \frac{1}{Re}L$, the Taylor series expansion of $A^{-1}$ is
\begin{equation}\label{eq:approx-A-inv}
    A^{-1} = \Delta t \sum\limits_{i=0}^{\infty} \left(-1\right)^{n}\left(\frac{\alpha\Delta t}{Re}\right)^n L^n
\end{equation}
And similarly, we can reformat $\bar{A}$ to a structure of $\frac{1}{\Delta t}I + S$ so that the inverse of $\bar{A}$ has a Taylor series expansion of
\begin{equation}\label{eq:approx-A-bar-inv}
    \bar{A}^{-1} = \Delta t \sum\limits_{i=0}^{\infty}\left(-1\right)^{n}\Delta t^n S^n
\end{equation}
where
\begin{equation}
    S = \begin{bmatrix}
        \frac{\alpha \Delta t}{Re}L & H \\
        E & -\frac{1}{\Delta t}I
    \end{bmatrix}
\end{equation}

In order to speed up the solution process, in the solution workflow, we can replace $A^{-1}$ and $\bar{A}^{-1}$ with truncated Taylor series expansions.
The most common approach is to use only the first-order approximation, i.e., use only the first term in the series expansions.
That is, $A^{-1} \approx \Delta t I$ and $\bar{A}^{-1} \approx \Delta t I$, where $I$ denotes the identity matrix with a corresponding dimension. ($A$ and $\bar{A}$ have different dimensions.)
Further, under the first-order approximation, the submatrices $\bar{A}_{00}^{-1}$ and $\bar{A}_{10}^{-1}$ can be found to be $\Delta t I$ and $0$.

The following is the actual solution workflow under the first-order approximation:
\begin{subequations}\label{eq:dibpm-first-order}
    \begin{align}
        & A u^{* *} = \vec{\gamma}^n + \vec{bc}_1\label{eq:dibpm-order2-1} \\
        & \Delta t EH \vec{f}^{n+1} = E \vec{u}^{* *} - \vec{u}_\Gamma^{n+1}\label{eq:dibpm-order2-2} \\
        & \vec{u}^* = \vec{u}^{* *} - \Delta t H \vec{f}^{n+1}\label{eq:dibpm-order2-3}\\
        & \Delta t DG \vec{p}^{n+1} = D \vec{u}^* - \vec{bc}_2\label{eq:dibpm-order2-4} \\
        & \vec{u}^{n+1} = \vec{u}^* - \Delta t G \vec{p}^{n+1}\label{eq:dibpm-order2-5}
    \end{align}
\end{subequations}

Using higher-order approximations for $A^{-1}$ and $\bar{A}^{-1}$ is achievable through the Taylor series expansions in \eqref{eq:approx-A-inv} and \eqref{eq:approx-A-bar-inv}.
However, the resulting linear systems in \eqref{eq:dibpm-order1-2} and \eqref{eq:dibpm-order1-4} may not be sparse due to the exponential terms in the Taylor series expansions.

In this section, we only described the specific solver (decoupled IBPM) we used to create the baseline data for benchmarking PINNs.
For other available solvers in PetIBM, please refer to the documentation \cite{chuang_petibm:_2018}.
% vim:ft=tex
