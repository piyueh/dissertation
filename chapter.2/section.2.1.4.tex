%! TEX root = main.tex

PetIBM implements schemes derived from~\cite{Perot1993}, including~\cite{Taira2007} and~\cite{Li2016}.
We presented the numerical schemes and implementation details of PetIBM in~\cite{chuang_petibm:_2018}.
Interested readers can also find verification and validation of PetIBM at reference~\cite{mesnard_petibm_nodate}.
Here we emphasize the performance gain of PetIBM after applying AmgXWrapper.

Figure~\ref{fig:petibm-speedups} shows the performance speedups comparing pure-CPU and CPU-GPU-mixed simulations.
In pure-CPU simulations, PetIBM solves velocity systems with the GAMG preconditioners and pressure systems with BoomerAMG preconditioners.
On the other hand, for CPU-GPU mixed simulations, the preconditioners for velocity and pressure are block-Jacobi and classical multigrid from AmgX, respectively.
The underlying linear system solvers are the biconjugate gradient stabilized solver from either PETSc or AmgX.
All other calculations, including solving the force systems, are done on CPUs.
The two subplots, figure~\ref{fig:petibm-speedups-small} and~\ref{fig:petibm-speedups-large}, represent the results using two different clusters.
Results from both clusters agree with each other.
We can see that a 4-GPU node or 2 2-GPU nodes are equivalent roughly to 5 pure-CPU nodes.
This again proves that adopting GPU computing in PetIBM can reduce both the time and monetary cost of CFD simulations.

% vim:ft=tex
